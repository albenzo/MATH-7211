\documentclass[10pt]{article}
\usepackage[utf8]{inputenc}
\usepackage{amscd}
\usepackage{amsmath}
\usepackage{amssymb}
\usepackage{amsthm}
\usepackage{listings}
\usepackage{enumerate}

\textwidth=15cm \textheight=22cm \topmargin=0.5cm \oddsidemargin=0.5cm \evensidemargin=0.5cm

\newcommand{\sk}{\vskip 10mm}
\newcommand{\bb}[1]{\mathbb{#1}}
\newcommand{\ra}{\rightarrow}

\theoremstyle{plain}
\newtheorem{problem}{Problem}
\newtheorem{lemma}{Lemma}[problem]

\theoremstyle{remark}
\newtheorem{tpart}{}[problem]
\newtheorem*{ppart}{}

\begin{document}

\begin{problem}
  
\end{problem}

\begin{proof}
  First note that if $15^\circ$ can be trisected then so can $30^\circ$ as we could bisect
  $30^\circ$, trisect $15^\circ$ and then double the resulting angle. As such it will
  suffice to show that we cannot trisect $15^\circ$.

  A number is constructible if, and only if, both its real and imaginary parts are
  constructible. If $15^\circ$ were constructible then so would $e^{i\cdot 10^\circ}$ as it would
  be the intersection of the angle and the unit circle. The real part of which is
  \[\alpha=\cos10^\circ = \frac{1}{2}\sqrt{\frac{1}{2}\left(4+2\cdot\left(\frac{1}{2}(1+i\sqrt{3})^{-\frac{1}{3}}+2^{\frac{2}{3}}(1+i\sqrt{3})\right)\right)} \]

  We know that a number is constructible if, and only if, we have an ascending
  chain of fields $\bb{Q}=F_0\subset\cdots\subset F_n=\bb{Q}[\alpha]$ where all of the intermediate degrees are two.
  This enforces that the degree of the extension must be a power of $2$. However
  for $\alpha$ at some point we will have to adjoin $(1+i\sqrt{3})^{-\frac{1}{3}}$ for which the
  extension will be of degree $3$. By the tower theorem this means that $3|\bb{Q}[\alpha]$ but
  this cannot occur.

  Therefore $\alpha$ is not constructible and it then follows that neither $15^\circ$ nor $30^\circ$ can
  be trisected.
\end{proof}

\sk

\begin{problem}
  
\end{problem}

\begin{itemize}
\item[(a)]
  Let $f(x)=x^2-x+1$. This polynomial has $\xi$ as a root. Moreover it is irreducible by
  the rational roots theorem as $\pm 1$ are not roots.
\item[(b)]
  The roots of $f$ are $\xi$ and $-e^{2\pi i/3}=-\xi^2$. Thus $\bb{Q}[\xi,-\xi^2]=\bb{Q}[\xi]$ is the
  splitting field for $f$.
\item[(c)]
  Since the degree of $f$ is 2 it follows that $[\bb{Q}[\xi]:\bb{Q}]=2$.
\end{itemize}

\sk

\begin{problem}
  
\end{problem}

\begin{itemize}
\item[(a)] The roots of $f(x)=x^4+1$ are 
\item[(b)] The roots of $g(x)=x^4+4$ are
\item[(c)] The roots of $p(x)=fg(x)=(x^4+1)(x^4+4)$ are
\item[(d)] The roots of $q(x)=(x^4-1)(x^4+4)$ are
\end{itemize}

\sk

\begin{problem}
  
\end{problem}

\begin{itemize}
\item[(a)] It will follow that $[\bb{Q}(\alpha):\bb{Q}]=4$ if
  $f(x)=x^4-4x^2+2$ is irreducible since $f(\alpha)=0$. It has
  no rational roots by the rational root theorem however
  we must now show that it cannot be the product of two
  irreducible quadratics.
\item[(b)]
\end{itemize}

\sk

\begin{problem}
  
\end{problem}

\begin{proof}
  
\end{proof}

\sk

\begin{problem}
  
\end{problem}

\begin{proof}
  
\end{proof}

\sk

\begin{problem}
  
\end{problem}

\begin{proof}
  
\end{proof}

%%%%%%%%%%%%%%%%%%%%%%%%%%%%%%%%%%%%%%%%%%%%%%%%%%%%%%%%%%%%%%%%%%%%%%%%%%%%%
\end{document}
