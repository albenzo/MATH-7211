\documentclass[10pt]{article}
\usepackage[utf8]{inputenc}
\usepackage{amscd}
\usepackage{amsmath}
\usepackage{amssymb}
\usepackage{amsthm}
\usepackage{listings}
\usepackage{enumerate}

\textwidth=15cm \textheight=22cm \topmargin=0.5cm \oddsidemargin=0.5cm \evensidemargin=0.5cm

\newcommand{\sk}{\vskip 10mm}
\newcommand{\bb}[1]{\mathbb{#1}}
\newcommand{\ra}{\rightarrow}

\theoremstyle{plain}
\newtheorem{problem}{Problem}
\newtheorem{lemma}{Lemma}[problem]

\theoremstyle{remark}
\newtheorem{tpart}{}[problem]
\newtheorem*{ppart}{}

\begin{document}

\begin{problem}
  Show that an angle of $30^\circ$ and an angle of $15^\circ$ cannot be trisected.
\end{problem}

\begin{proof}
  First note that if $15^\circ$ can be trisected then so can $30^\circ$ as we could bisect
  $30^\circ$, trisect $15^\circ$ and then double the resulting angle. As such it will
  suffice to show that we cannot trisect $15^\circ$.

  A number is constructible if, and only if, both its real and imaginary parts are
  constructible. If $15^\circ$ were constructible then so would $e^{i\cdot 10^\circ}$ as it would
  be the intersection of the angle and the unit circle. The real part of which is
  \[\alpha=\cos10^\circ = \frac{1}{2}\sqrt{\frac{1}{2}\left(4+2\cdot\left(\frac{1}{2}(1+i\sqrt{3})^{-\frac{1}{3}}+2^{\frac{2}{3}}(1+i\sqrt{3})\right)\right)} \]

  We know that a number is constructible if, and only if, we have an ascending
  chain of fields $\bb{Q}=F_0\subset\cdots\subset F_n=\bb{Q}[\alpha]$ where all of the intermediate degrees are two.
  This enforces that the degree of the extension must be a power of $2$. However
  for $\alpha$ at some point we will have to adjoin $(1+i\sqrt{3})^{-\frac{1}{3}}$ for which the
  extension will be of degree $3$. By the tower theorem this means that $3|\bb{Q}[\alpha]$ but
  this cannot occur.

  Therefore $\alpha$ is not constructible and it then follows that neither $15^\circ$ nor $30^\circ$ can
  be trisected.
\end{proof}

\sk

\begin{problem}
  Let $\xi = e^{2\pi i /6} = \cos (2\pi/6) + i \sin (2\pi/6)$ be a primitive
  $6^{th}$ root of unity over $\mathbb{Q}$. Find each of the following:
\begin{enumerate}
    \item The minimum polynomial $f(x) \in \mathbb{Q}[x]$ of $\xi$ over $\mathbb{Q}$.
    \item The splitting field $F$ of $f(x)$ over $\mathbb{Q}$.
    \item $[F:\mathbb{Q}]$.
\end{enumerate}
\end{problem}

\begin{itemize}
\item[(a)]
  Let $f(x)=x^2-x+1$. This polynomial has $\xi$ as a root. Moreover it is irreducible by
  the rational roots theorem as $\pm 1$ are not roots.
\item[(b)]
  The roots of $f$ are $\xi$ and $-e^{2\pi i/3}=-\xi^2$. Thus $\bb{Q}[\xi,-\xi^2]=\bb{Q}[\xi]$ is the
  splitting field for $f$.
\item[(c)]
  Since the degree of $f$ is 2 it follows that $[\bb{Q}[\xi]:\bb{Q}]=2$.
\end{itemize}

\sk

\begin{problem}
  Find a splitting field extension $K: \mathbb{Q}$ for each
  of the following polynomials over $\mathbb{Q}$ and in each
  case determine the degree $[K:\mathbb{Q}]$.
\begin{equation*}
    (a) \text{  } x^4 + 1 \hspace{5mm} (b) \text{  } x^4 + 4 \hspace{5mm}
    (c) \text{  } (x^4 +1)(x^4+4) \hspace{5mm} (d) \text{  } (x^4-1)(x^4+4)
\end{equation*}
\end{problem}

\begin{itemize}
\item[(a)] The roots of $f(x)=x^4+1$ are $r:=e^{\pi i/4},r^3,r^5,$ and $r^7$. Since
  each all of the other roots can be expressed as a power of $r$ we have that
  the splitting field of $f$ is $\bb{Q}[r,r^3,r^5,r^7]=\bb{Q}[r]$ the degree of which
  is $4$ as $f$ is irreducible and thus the minimal polynomial. The irreducibility can
  be checked by shifting to $f(x+1)$ and apply Eisenstein's Criterion with $p=2$.
\item[(b)] The roots of $g(x)=x^4+4$ are the same roots as above but with each multiplied
  by $\sqrt{2}$. Let $s:=\sqrt{2}e^{\pi i /4}$. Then the other roots are $s^3/2,s^5/4,$ and
  $s^7/8$. Similar to before the splitting field is then $\bb{Q}[s]$ and since this
  polynomial is irreducible we have that $[\bb{Q}[s]:s]=4$.

\item[(c)] The roots of $p(x)=fg(x)=(x^4+1)(x^4+4)$ are the roots of both part $a$ and $b$.
  Since the roots here are cyclic if we take $rs$ we get $\sqrt{2}e^{\pi i /2}$. Keep multiplying
  that by $r$ and we can hit every root. Thus the splitting field will be
  $\bb{Q}[r]$ and the minimal polynomial will be the one from part $(a)$ giving
  us that $[\bb{Q}[r]:\bb{Q}]=4$ for the degree of our splitting field.
\item[(d)] The roots of $q(x)=(x^4-1)(x^4+4)$ are the roots of part $b$ as well
  as $\pm 1$ and $\pm i$. However $s^2/2=i$ which means that we can express all of the roots
  in terms of $s$. Similar to part $(c)$ our splitting field is the same as $b$, $\bb{Q}[s]$.
  As before the degree of this splitting field is 4.
\end{itemize}

\sk

\begin{problem}
  Let $f(x) \in \mathbb{Q}[x]$ be the minimal polynomial of $\alpha = \sqrt{2 + \sqrt{2}}$.
\begin{enumerate}
    \item Show that $f(x) = x^4 - 4x^2 + 2$. Thus, $[\mathbb{Q}(\alpha): \mathbb{Q}] = 4$.
    \item Show that $\mathbb{Q}(\alpha)$ is the splitting field of $f(x)$ over $\mathbb{Q}$.
\end{enumerate}
\end{problem}

\begin{itemize}
\item[(a)] It will follow that $[\bb{Q}(\alpha):\bb{Q}]=4$ if
  $f(x)=x^4-4x^2+2$ is irreducible since $f(\alpha)=0$. However
  $f$ is irreducible by Eisenstein's criterion using $2$.
\item[(b)] The roots of $f$ are $\pm\sqrt{2\pm\sqrt{2}}$
\end{itemize}

\sk

\begin{problem}
  Let $\mathbb{F}_p = \mathbb{Z}/p\mathbb{Z}$ be the field with $p$
  elements, where $p$ is a prime number. Write down all monic cubic
  polynomials in $\mathbb{F}_2[x]$, factor them completely into
  irreducible factors and construct a splitting field for each
  of them. Which of these fields are isomorphic?
\end{problem}

\begin{proof}
  
\end{proof}

\sk

\begin{problem}
  Let $f(x) = x^3 + 2x + 2 \in \mathbb{F}_3[x]$.
\begin{enumerate}
    \item Show that $f(x)$ is irreducible in $\mathbb{F}_3[x]$.
    \item Let $\alpha$ be a root of $f(x)$ in some extension field $K$
      of $\mathbb{F}_3$, so that
      $[\mathbb{F}_3[\alpha]: \mathbb{F}_3] = \mathrm{deg} f(x) = 3$.
      Show that $\mathbb{F}_3[\alpha]$ is a splitting field of $f(x)$
      over $\mathbb{F}_3$.
\end{enumerate}
\end{problem}

\begin{proof}
  
\end{proof}

\sk

\begin{problem}
  Suppose that $f(x) \in F[x]$ is irreducible of degree $n > 0$,
  and let $L$ be the splitting field of $f(x)$ over $F$.
\begin{enumerate}
    \item Suppose that $[L:F]=n!$. Prove that $f(x)$ is irreducible.
    \item Show that the converse of part $(a)$ is false.
\end{enumerate}
\end{problem}

\begin{proof}
  
\end{proof}

%%%%%%%%%%%%%%%%%%%%%%%%%%%%%%%%%%%%%%%%%%%%%%%%%%%%%%%%%%%%%%%%%%%%%%%%%%%%%
\end{document}
