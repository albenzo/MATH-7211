\documentclass[10pt]{article}
\usepackage[utf8]{inputenc}
\usepackage{amscd}
\usepackage{amsmath}
\usepackage{amssymb}
\usepackage{amsthm}
\usepackage{listings}
\usepackage{enumerate}

\textwidth=15cm \textheight=22cm \topmargin=0.5cm \oddsidemargin=0.5cm \evensidemargin=0.5cm

\newcommand{\sk}{\vskip 10mm}
\newcommand{\bb}[1]{\mathbb{#1}}
\newcommand{\ra}{\rightarrow}

\theoremstyle{plain}
\newtheorem{problem}{Problem}
\newtheorem{lemma}{Lemma}[problem]

\theoremstyle{remark}
\newtheorem{tpart}{}[problem]
\newtheorem*{ppart}{}

\begin{document}

\begin{problem}
  Give an example of fields $F\subset E\subset K$   such that $K$ is a root
  extension of $F$ but $E$ is not a root extension of $F$.
  (Hint: Look at $K=\bb{Q}(\zeta)$ where $\zeta$ is the primitive
  7th root of unity).
\end{problem}

\begin{proof}
  
\end{proof}

\sk

\begin{problem}
  Let $F=\bb{Q}$, $E=\bb{Q}(\sqrt{3})$, $K=\bb{Q}(\sqrt{\sqrt{3}+1})$.
  Show that $E/F$ and $K/E$ are Galois extensions, but that $K/F$
  is not Galois. Find the minimal polynomial of $\sqrt{\sqrt{3}+1}$
  and find its Galois group.
\end{problem}

\begin{proof}
  
\end{proof}

\sk

\begin{problem}
  Find a root extension of $\bb{Q}$ containing the splitting fields
  of each of the following polynomials.
  \begin{enumerate}
  \item[(a)] $x^4+1$
  \item[(b)] $x^4+3x^2+1$
  \item[(c)] $x^5+4x^3+x$
  \item[(d)] $(x^3-2)(x^7-5)$
  \end{enumerate}
\end{problem}

\begin{proof}
  
\end{proof}

\sk

\begin{problem}
  Give an example of a polynomial $\bb{Q}[x]$ which is solvable by radicals,
  but whose splitting field is not a root extension of $\bb{Q}$.
\end{problem}

\begin{proof}
  
\end{proof}

\sk

\begin{problem}
  For $r$ a positive integer, define $f_r(x)\in\bb{Q}[x]$ by
  \[
    f_r(x)=(x^2+4)x(x^2-4)(x^2-16)\cdots(x^2-4r^2)
  \]
  \begin{enumerate}
  \item[(a)] Give a (rough) sketch of the graph of $f_r(x)$
  \item[(b)] Show that if $k$ is an odd integer, then $|f_r(k)|\geq 5$.
  \item[(c)] Show that $g_r(x)=f_r(x)-2$ is irreducible over $\bb{Q}$ and
    determine its Galois group when $2r+3=p$ is prime.
  \end{enumerate}
\end{problem}

\begin{proof}
  
\end{proof}

%%%%%%%%%%%%%%%%%%%%%%%%%%%%%%%%%%%%%%%%%%%%%%%%%%%%%%%%%%%%%%%%%%%%%%%%%%%%%
\end{document}
