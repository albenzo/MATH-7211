\documentclass[10pt]{article}
\usepackage[utf8]{inputenc}
\usepackage{amscd}
\usepackage{amsmath}
\usepackage{amssymb}
\usepackage{amsthm}
\usepackage{listings}
\usepackage{enumerate}

\textwidth=15cm \textheight=22cm \topmargin=0.5cm \oddsidemargin=0.5cm \evensidemargin=0.5cm

\newcommand{\sk}{\vskip 10mm}
\newcommand{\bb}[1]{\mathbb{#1}}
\newcommand{\ra}{\rightarrow}

\theoremstyle{plain}
\newtheorem{problem}{Problem}
\newtheorem{lemma}{Lemma}[problem]

\theoremstyle{remark}
\newtheorem{tpart}{}[problem]
\newtheorem*{ppart}{}

\begin{document}

\begin{problem}
  Give an example of fields $F\subset E\subset K$   such that $K$ is a root
  extension of $F$ but $E$ is not a root extension of $F$.
  (Hint: Look at $K=\bb{Q}(\zeta)$ where $\zeta$ is the primitive
  7th root of unity).
\end{problem}

\sk

\begin{problem}
  Let $F=\bb{Q}$, $E=\bb{Q}(\sqrt{3})$, $K=\bb{Q}(\sqrt{\sqrt{3}+1})$.
  Show that $E/F$ and $K/E$ are Galois extensions, but that $K/F$
  is not Galois. Find the minimal polynomial of $\sqrt{\sqrt{3}+1}$
  and find its Galois group.
\end{problem}

We can see that $E/F$ and $K/E$ are Galois since they are
both splitting fields of separable polynomials $x^2-3$ and
$x^2-(\sqrt{3}+1)$ respectively.

Now we will show that the $K/F$ is not Galois. First note that the
minimal polynomial of $\sqrt{\sqrt{3}+1}$ is $x^4-2x^2-2$. If it
was Galois then it would contain all of the roots which are
$\pm\sqrt{1\pm\sqrt{3}}$. However then it would contain the product
\[
  \sqrt{1+\sqrt{3}}\sqrt{1-\sqrt{3}}=i\sqrt{2}
\]

However $i\notin K$ which shows that $K/E$ is not Galois.

To find the Galois group of $f(x)=x^4-2x^2-2$ we will follow the procedure
in Dummit and Foote beginning on page 615. The resolvent cubic for
$f$ is $h(x)=x^3+4x^2+12x$. Since $h$ is reducible into a linear term
and a quadratic the Galois group is either $\bb{Z}_4$ or $D_8$. However
since the discriminant of $f$ is $-4608$ which is negative. As such
the Galois group cannot be cyclic. Therefore $\text{Gal}(x^4-2x^2-2)\cong D_8$.
\sk

\begin{problem}
  Find a root extension of $\bb{Q}$ containing the splitting fields
  of each of the following polynomials.
  \begin{enumerate}
  \item[(a)] $x^4+1$
  \item[(b)] $x^4+3x^2+1$
  \item[(c)] $x^5+4x^3+x$
  \item[(d)] $(x^3-2)(x^7-5)$
  \end{enumerate}
\end{problem}
\sk

\begin{enumerate}
\item [(a)] The roots of $x^4+1$ are $e^{\pi i/4},e^{3\pi i/4},e^{5\pi i/4}$, and $e^{7\pi i/4}$.
  Since these are all roots of $x^4-(-1)$ the field
  $\bb{Q}[e^{\pi i/4},e^{3\pi i/4},e^{5\pi i/4},e^{7\pi i/4}]$ has a root tower from adjoining
  each one in turn.
\item [(b)] The roots of $x^4+3x^2+1$ are $\pm\sqrt{\frac{-3\pm\sqrt{5}}{2}}$. The splitting
  field will then be contained in the root tower where we adjoint $\sqrt{5}$ followed
  by $\sqrt{3+\sqrt{5}}$ and finally $i$.
\item [(c)] The roots of $x^5+4x^3+x$ are $0,\pm\sqrt{-2\pm\sqrt{3}}$. Similar to part (b)
  we can get a root tower containing the splitting field by adjoining
  $\sqrt{3},\sqrt{2+\sqrt{3}}$, and $i$ in order.
\item [(d)] The roots of $(x^3-2)(x^7-5)$ are $\sqrt{2}\zeta_3^i,\sqrt{5}\zeta_7^j$ for $0\leq i<3$ and
  $0\leq j<7$. We can then obtain a root extension containing the splitting field by
  adjoining each of the roots in turn.
\end{enumerate}

\sk

\begin{problem}
  Give an example of a polynomial $\bb{Q}[x]$ which is solvable by radicals,
  but whose splitting field is not a root extension of $\bb{Q}$.
\end{problem}


\sk

\begin{problem}
  For $r$ a positive integer, define $f_r(x)\in\bb{Q}[x]$ by
  \[
    f_r(x)=(x^2+4)x(x^2-4)(x^2-16)\cdots(x^2-4r^2)
  \]
  \begin{enumerate}
  \item[(a)] Give a (rough) sketch of the graph of $f_r(x)$
  \item[(b)] Show that if $k$ is an odd integer, then $|f_r(k)|\geq 5$.
  \item[(c)] Show that $g_r(x)=f_r(x)-2$ is irreducible over $\bb{Q}$ and
    determine its Galois group when $2r+3=p$ is prime.
  \end{enumerate}
\end{problem}

\begin{proof}
  \begin{enumerate}
  \item[(a)] Will have roots at $\pm 2r$ then go off on its own.
  \item[(b)] 
  \item[(c)] First note that for $f_r(x)$ is monic and that the coeffiecient
    for every other term will be divisible by 2 as $4r^2$ is even. Moreover
    as we multiply by $x$ in the product form the polynomial has no constant
    term. Thus $g_r(x)=f_r(x)-2$ is irreducible by Eisenstein's criterion with
    2.

    Since $|f_r(k)|\geq 5$ for odd $k$ we know that $g_r$ will have the same
    number of roots as $f_r$. We also know that there are two complex
    roots. As such one of the automorphisms in our Galois group will
    be a transposition supplied by complex conjugation. In addition since $g_r$
    is irreducible the action of the Galois group on the roots is transitive.
    Since the degree of $g_r$ is prime we have that $\text{Gal}(g_r)\cong S_p$
    as the group is transitive and contains a transposition.
  \end{enumerate}
\end{proof}

%%%%%%%%%%%%%%%%%%%%%%%%%%%%%%%%%%%%%%%%%%%%%%%%%%%%%%%%%%%%%%%%%%%%%%%%%%%%%
\end{document}
