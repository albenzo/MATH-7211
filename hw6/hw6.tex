\documentclass[10pt]{article}
\usepackage[utf8]{inputenc}
\usepackage{amscd}
\usepackage{amsmath}
\usepackage{amssymb}
\usepackage{amsthm}
\usepackage{listings}
\usepackage{enumerate}
\usepackage[all,cmtip]{xy}

\textwidth=15cm \textheight=22cm \topmargin=0.5cm \oddsidemargin=0.5cm \evensidemargin=0.5cm

\newcommand{\sk}{\vskip 10mm}
\newcommand{\bb}[1]{\mathbb{#1}}
\newcommand{\ra}{\rightarrow}
\newcommand{\wt}[1]{\widetilde{#1}}

\theoremstyle{plain}
\newtheorem{problem}{Problem}
\newtheorem{lemma}{Lemma}[problem]

\theoremstyle{remark}
\newtheorem{tpart}{}[problem]
\newtheorem*{ppart}{}

\begin{document}

\begin{problem}[7.26]
  Let $M$ and $N$ be finitely generated $R$-modules over a PID $R$. Compute
  $M\otimes_R N$. As a special case, if $M$ is a finite abelian group with
  invariant factors $s_1,\ldots,s_t$ (where as usual we assume that $s_i$
  divides $s_{i+1}$), show that $M\otimes_{\bb{Z}}M$ is a finite group of order
  $\prod_{j=1}^ts_j^{2t-2j+1}$.
\end{problem}

\begin{proof}
  By the structure theorem for finitely generated modules over a PID we
  have that
  \[
    M\cong R^r\oplus R/(a_i)\oplus\cdots \oplus R/(a_m)
  \]
  and
  \[
    N\cong R^s\oplus R/(b_1)\oplus \cdots \oplus R/(b_n)
  \]
  where $a_i|a_{i+1}$ and $b_i|b_{i+1}$
  Then by Theorem 2.18 of Adkins' book the tensor $M\otimes_R N$ distributes over
  the direct sum. Together with $R/(a_i)\otimes_R R/(b_j)\cong R/(a_i+b_j)$ we get that
  \[
    M\otimes_R N\cong \bigoplus_{0<i\leq m,0<j\leq n}\left(R/(a_i+b_j)\right)\bigoplus_{0<i\leq m}\left(R/(a_i)\otimes_R R^t\right)\bigoplus_{0<j\leq n}\left(R^s\otimes_R R/(b_j)\right)
  \]


  If we let $M$ be a finite abelian group with invariant factor decomposition
  \[
    M\cong \bigoplus_{0<i\leq t}\bb{Z}_{s_i}
  \]

  Then, from above, the tensor of $M$ with itself is
  \[
    M\otimes_{\bb{Z}} M \cong \bigoplus_{0<i,j\leq t}\bb{Z}_{\gcd(s_i,s_j)}
  \]

  However since $s_i|s_j$ for $i\leq j$ we have that $\gcd(s_i,s_j)$. This will
  cause $2t-1$ copies of $\bb{Z}_{s_1}$ to appear, $2t-3$ for $\bb{Z}_{s_2}$, and so on until
  we have only a single copy of $\bb{Z}_{s_t}$. Thus $M\otimes_{\bb{Z}}M$ will have
  order $\prod_{j=1}^ts_j^{2t-2j+1}$.
\end{proof}

\sk

\begin{problem}[7.30]
  \begin{enumerate}
  \item[(a)] Let $F$ be a field and $K$ a field containing $F$.
    If $f(X)\in F[X]$, show that there is an isomorphism of $K$-algebras:
    \[
      K\otimes_F(F[X]/\langle f(X)\rangle\cong K[X]/\langle f(X)\rangle)
    \]
  \item[(b)] By choosing $F, f(x)$, and $K$ appropriately, find an
    example of two fields $K$ and $L$ containing $F$ such that the
    $F$-algebra $K\otimes_F L$ has nilpotent elements.
  \end{enumerate}
\end{problem}

\begin{proof}
  \begin{enumerate}
  \item[(a)] Define a map $\varphi:K\times F[x]/\langle g(x)\rangle\rightarrow K[x]/\langle g(x)\rangle$ via
    \[
      \varphi(k,f)= kf
    \]
    Now briefly verify that $\varphi$ is $F$-middle linear
    \begin{align*}
      \varphi(ak,fb) &= akfb &= a\varphi(k,f)b\\
      \varphi(k_1+k_2,f)&= k_1 f+k_2 f &= \varphi(k_1,f)+\varphi(k_2,f)\\
      \varphi(k,f_1+f_2)&= kf_1+kf_2 &= \varphi(k,f_1)+\varphi(k,f_2)\\
      \varphi(ka,f) &= kaf &= \varphi(k,af)\\
    \end{align*}

    Since $\varphi$ is $F$-middle linear it induces a map
    $\wt{\varphi}:K\otimes_F F[x]/\langle g(x)\rangle\rightarrow K[x]/\langle g(x)\rangle$. Now define
    a map $\psi:K[x]/\langle g(x)\rangle\rightarrow K\otimes_F F[x]/\langle g(x)\rangle$ by $\psi(kx^i)=k\otimes x^i$ and
    extending linearly. We will now show that $\wt{\varphi}$ and $\psi$ are inverses.

    First $\wt{\varphi}\circ\psi$
    \begin{align*}
      \wt{\varphi}\circ\psi(\sum a_ix^i) &= \wt{\varphi}(\sum a_i \otimes x^i)\\
                        &= \sum a_i x^i\\
    \end{align*}

    Then for $\psi\circ\wt{\varphi}$
    \begin{align*}
      \psi\circ\wt{\varphi}(k\otimes\sum a_i x^i) &= \psi\circ\wt{\varphi}(\sum k\otimes a_i x^i)\\
                            &= \psi(\sum\wt{\varphi}(k\otimes a_i x^i))\\
                            &= \psi(\sum ka_i x^i)\\
                            &= \sum \psi(ka_ix^i)\\
                            &= \sum ka_i\otimes x^i\\
                            &= \sum k\otimes a_i x^i\\
    \end{align*}
    Since $\wt{\varphi}$ has an inverse it is indeed an isomorphism. Which
    shows that $K\otimes_F F[x]/\langle g(x)\rangle$ is isomorphic to $K[x]/\langle g(x)\rangle$.
  \item[(b)] Let $F$ be rational functions of $t$ with coefficients in $\bb{Z}_2$.
    Then let $K$ be the splitting field of $x^2-t$. Then in $K\otimes_F K$ the
    element $t^{1/2}\otimes 1+1\otimes t^{1/2}$ is zero when squared making it an idempotent.
  \end{enumerate}
\end{proof}

\sk


\begin{problem}[7.31]
  Let $F$ be a field. Show that $F[X,Y]\cong F[X]\otimes_F F[Y]$ where the
  isomorphism is an isomorphism of $F$-algebras.
\end{problem}

\begin{proof}
  Define $\varphi:F[x]\times F[y]\rightarrow F[x]\times F[y]$ via
  \[
    \varphi(\sum_i a_i x^i,\sum_j b_j x^j) = \sum_{i,j}a_i b_j x^i y^j
  \]
  Since we are in a polynomial ring over a field it follows that
  $\varphi$ is $F$-middle linear. As such there is a map
  $\wt{\varphi}:F[x]\otimes F[y]\rightarrow F[x,y]$ that mimics $\varphi$.

  Now define $\psi:F[x,y]\rightarrow F[x]\otimes F[y]$ via $\psi(cx^i y^j)=c(x^i\otimes y^j)$ and
  extending linearly. We will now show that $\wt{\varphi}$ and $\psi$ are
  inverses. Since both of these maps are linear we only need to
  verify them on either $cx^i y^j$ and $a_i x^i\otimes b_j y^j$.

  First for $\wt{\varphi}\circ \psi$ we have
  \begin{align*}
    \wt{\varphi}\circ\psi(cx^i y^j) &= \wt{\varphi}(c(x^i\otimes y^j))\\
                     &= \wt{\varphi}(cx^i\otimes y^j)\\
                     &= cx^i y^j\\
  \end{align*}

  Then for $\psi\circ\wt{\varphi}$
  \begin{align*}
    \psi\circ\wt{\varphi}(a_i x^i \otimes b_j y^j) &= \psi (a_i b_j x^i y^j)\\
                             &= a_i b_j (x^i\otimes y^j)\\
                             &= a_i x^i \otimes b_j y^j\\
  \end{align*}

  Thus $\wt{\varphi}$ is an isomorphism which shows that $F[x,y]$ is isomorphic
  to $F[x]\otimes F[y]$.
\end{proof}

\sk

\begin{problem}[7.34]
  Let $F$ be a field, $V$ and $W$ finite-dimensional vector spaces over $F$,
  and let $T\in\text{End}_F(V),S\in\text{End}_F(W)$.
  \begin{enumerate}
  \item[(a)] If $\alpha$ is an eigenvalue of $S$ and $\beta$ is an eigenvalue of
    $T$, show that the product $\alpha\beta$ is an eigenvalue of $S\otimes T$.
  \item[(b)] If $S$ and $T$ are diagonalizable, show that $S\otimes T$
    is diagonalizable.
  \end{enumerate}
\end{problem}

\begin{proof}
  \begin{enumerate}
  \item[(a)] Let $u,v$ be eigenvectors with eigenvalues $\alpha,\beta$ for linear
    transformations $S$ and $T$ respectively. Then
    \[
      S\otimes T(u\otimes v) = S(u)\otimes T(v)= \alpha u\otimes \beta v = \alpha\beta (u\otimes v)
    \]
  \item[(b)] Since $S$ and $T$ are diagonalizable they each have a number
    of eigenvalues equal to the dimension of the space they act on.
    By the previous problem the product of two eigenvalues gives an eigenvalue
    which means there will be $\dim(V)\dim(W)=\dim(V\otimes W)$ eigenvalues
    for $S\otimes T$ making it diagonalizable.
  \end{enumerate}
\end{proof}

\sk

\begin{problem}[10.4.3]
  Show that $\bb{C}\otimes_{\bb{R}}\bb{C}$ and $\bb{C}\otimes_{\bb{C}}\bb{C}$
  are both left $\bb{R}$ modules but are not isomorphic as $\bb{R}$-modules.
\end{problem}

\begin{proof}
  The complex numbers can be given the bimodule structures
  $(\bb{R},\bb{R}),(\bb{R},\bb{C})$, and $(\bb{C},\bb{R})$ by left
  and right multiplication from the correct field. As such both of the
  above tensors can be given be given a left $\bb{R}$-module structure.

  To show they are not isomorphic first note that from Dummit  10.4.19
  we know that $\bb{C}\otimes_{\bb{C}}\bb{C}\cong\bb{C}$ which has rank $2$ as
  a free $\bb{R}$ module. Meanwhile $\bb{C}\otimes_{\bb{R}}\bb{C}$ has free
  rank $4$ as a left $\bb{R}$-module (Dummit pg. 375) which implies that they cannot
  be isomorphic.
\end{proof}

\sk

\begin{problem}[10.4.5]
  Let $A$ be a finite abelian group of order $n$ and let $p^k$ be the largest
  power of a prime $p$ dividing $n$. Prove that
  $\bb{Z}/p^k\bb{Z}\otimes_{\bb{Z}}A$ is isomorphic to the Sylow $p$-subgroup
  of $A$.
\end{problem}

\begin{proof}
  Let $Syl_p(A)$ denote the Sylow $p$-subgroup of $A$. Then by theorem
  2.18 from Adkins' we have that
  \[
    \bb{Z}_{p^k}\otimes_{\bb{Z}} A \cong \bigoplus_{q\ prime}\bb{Z}_{p^k}\otimes_{\bb{Z}} Syl_q(A)
  \]
  Now we will show that $\bb{Z}_{p^k}\otimes_{\bb{Z}}Syl_q(A)$ is zero when
  $q\neq p$ and is isomorphic to $Syl_p(A)$ otherwise.

  Let's start when $p\neq q$. Then let $q^l$ be the highest power of
  $q$ that divides $n$. As $p$ and $q$ are relatively prime there
  exist $\alpha$ and $\beta$ such that $\alpha p^k+\beta q^l=1$. Then given
  $x\otimes a\in \bb{Z}_{p^k}\otimes_{\bb{Z}} Syl_q(A)$ we have
  \[
    x\otimes a = x(\alpha p^k+\beta q^l)\otimes a = (x\alpha p^k+x\beta q^l)\otimes a = x\otimes q^l a= x\otimes 0 = x \otimes p^k0 = x p^k\otimes 0 = 0\otimes 0
  \]
  which shows that every element of $\bb{Z}_{p^k}\otimes_{\bb{Z}} Syl_q(A)$ is
  trivial making the group itself trivial.

  The fact that $\bb{Z}_{p^k}\otimes_{\bb{Z}} Syl_p(A)\cong Syl_p(A)$ follows from
  both constituents have size $p^k$. As such the inner action of $\bb{Z}$
  is equivalent to the corresponding action by $\bb{Z}_{p^k}$. Thus
  $\bb{Z}_{p^k}\otimes_{\bb{Z}} Syl_p(A)$ is isomorphic to
  $\bb{Z}_{p^k}\otimes_{\bb{Z}_{p^k}} Syl_p(A)$ which is then isomorphic to $Syl_p(A)$.

  Therefore $\bb{Z}_{p^k}\otimes_{\bb{Z}} A$ is isomorphic to the Sylow $p$-subgroup of $A$.
\end{proof}

\sk

\begin{problem}[10.4.10]
  Suppose $R$ is commutative and $N\cong R^n$ is a free $R$-module of rank
  $n$ with $R$-module basis $e_1,\ldots,e_n$.
  \begin{enumerate}
  \item[(a)] For any nonzero $R$-module $M$ show that every element
    of $M\otimes N$ can be written uniquely in the form
    $\sum_{i=1}^n m_i\otimes e_i$ where $m_i\in M$. Deduce that if
    $\sum_{i=1}^nm_i\otimes e_i=0$ in $M\otimes N$ then $m_i=0$ for
    $i=1,\ldots,n$.
  \item[(b)] Show that if $\sum m_i\otimes n_i=0$ in $M\otimes N$ where the
    $n_i$ are merely assumed to be $R$-linearly independent then it is not
    necessarily true that all the $m_i$ are 0. [Consider
    $R=\bb{Z},n=1,M=\bb{Z}/2\bb{Z}$, and the element $1\otimes 2$.]
  \end{enumerate}
\end{problem}

\begin{proof}
  \begin{enumerate}
  \item[(a)] Let $m\otimes n\in M\otimes_R N$. Then since $N$ is free we can rewrite $m\otimes n$ as
    \[
      m\otimes n = m \otimes \left(\sum_1^n a_i e_i\right) = \sum_i^n m\otimes a_i e_i= \sum_1^n a_i m \otimes e_i
    \]
    Since this process is entirely reversible if another element had the same
    decomposition then it would be equal to the original value. Moreover the decomposition
    in terms of $e_i$ on the right are unique. Therefore the decomposition is unique.
    As such we know that $\sum_1^n 0\otimes e_i=0 $ and by uniqueness it follows that this is the
    only way of writing zero.
  \item[(b)] Let $R=\bb{Z},n=1,M=\bb{Z}_2$, and $N=\bb{Z}$. Then $1\otimes 2=1\cdot 2\otimes 1=0\otimes 1=0$
    which fulfills the conditions lain out above.
  \end{enumerate}
\end{proof}

\sk

\begin{problem}[10.4.24]
  Prove that the extension of scalars from $\bb{Z}$ to the Gaussian integers
  $\bb{Z}[i]$ of the ring $\bb{R}$ is isomorphic to $\bb{C}$ as a ring:
  $\bb{Z}[i]\otimes_{\bb{Z}}\bb{R}\cong\bb{C}$ as rings.
\end{problem}

\begin{proof}
  Let $\varphi:\bb{Z}[i]\times\bb{R}\rightarrow \bb{C}$ be defined via $\varphi(k,x)=kx$. Since it
  is effectively the same map that we defined before we can see that it
  is $\bb{Z}$-middle linear. As such there is an induced map
  $\wt{\varphi}:\bb{Z}[i]\otimes_{\bb{Z}}\bb{R}\rightarrow\bb{C}$ where $\wt{\varphi}(k\otimes x)=kx$.
  Define $\psi:\bb{C}\rightarrow \bb{Z}[i]\otimes_{\bb{Z}}\bb{R}$ via $\psi(x)=1\otimes x$ and $\psi(ix)=i\otimes x$
  when $x\in \bb{R}$ then extending linearly. We will now show that $\wt{\varphi}$ and
  $\psi$ are inverses.

  First for $\psi\circ\wt{\varphi}$ we have
  \begin{align*}
    \psi\circ\wt{\varphi}((a+bi)\otimes x) &= \psi(ax+bxi)\\
                        &= \psi(ax)+\psi(bxi)\\
                        &= 1\otimes ax + i\otimes bx\\
                        &= a\otimes x + bi\otimes x\\
                        &= (a+bi)\otimes x
  \end{align*}

  Then for $\wt{\varphi}\circ\psi$
  \begin{align*}
    \wt{\varphi}\circ\psi(u+iv) &= \wt{\varphi}(1\otimes u+i\otimes v)\\
                   &= \wt{\varphi}(1\otimes u)+\wt{\varphi}(i\otimes v)\\
                   &= u+iv
  \end{align*}

  Thus $\psi$ and $\wt{\varphi}$ are inverses which makes $\wt{\varphi}$ an isomorphism
  which shows that $\bb{C}$ is isomorphic to $\bb{Z}[i]\otimes \bb{R}$.
\end{proof}

\sk

\begin{problem}[10.4.27]
  \begin{enumerate}
  \item[(a)] Write down a formula for the multiplication of two elements
    $a\cdot 1+b\cdot e_2+c\cdot e_3+d\cdot e_4$ and
    $a'\cdot 1+b'\cdot e_2+c'\cdot e_3+d'\cdot e_4$ in the example
    $A=\bb{C}\otimes_{\bb{R}}\bb{C}$ following proposition 21
    (where $1=1\otimes 1$ is the identity of $A$).
  \item[(b)] Let $\epsilon_1=\frac{1}{2}(1\otimes 1+i\otimes i)$ and
    $\epsilon_2=\frac{1}{2}(1\otimes 1-i\otimes i)$. Show that
    $\epsilon_1\epsilon_2=0,\epsilon_1+\epsilon_2=1$, and
    $\epsilon_j^2=\epsilon_j$ for $j=1,2$ ($\epsilon_1$ and $\epsilon_2$ are
    called orthogonal idempotents in $A$). Deduce that $A$ is isomorphic
    as a ring to the direct product of two principle ideals:
    $A\cong A\epsilon_1\times A\epsilon_2$ (cf. Exercise 1, Section 7.6).
  \item[(c)] Prove that the map
    $\varphi:\bb{C}\otimes\bb{C}\rightarrow\bb{C}\times\bb{C}$ by
    $\varphi(z_1,z_2)=(z_1z_2,z_1\bar{z_2})$, where $\bar{z_2}$
    denotes the complex conjugate of $z_2$, is an $\bb{R}$-bilinear map.
  \item[(d)] Let $\Phi$ be the $\bb{R}$-module homomorphism from $A$ to
    $\bb{C}\times\bb{C}$ obtained from $\varphi$ in (c). Show that
    $\Phi(\epsilon_1)=(0,1)$ and $\Phi(\epsilon_2)=(1,0)$. Show also that $\Phi$
    is $\bb{C}$-linear, where the action of $\bb{C}$ is on the left tensor
    factor in A and on both factors in $\bb{C}\times\bb{C}$. Deduce that
    $\Phi$ is surjective. Show that $\Phi$ is a $\bb{C}$-algebra isomorphism.
  \end{enumerate}
\end{problem}

\begin{proof}
  \begin{enumerate}
  \item[(a)] Let $e_1=1\otimes 1, e_2=1\otimes i, e_3= i\otimes 1, e_4 = i\otimes i. $. Then multiply out to get
    \[
      a a' e_{1}^{2} + a' b e_{1} e_{2} + a b' e_{1} e_{2} + b b' e_{2}^{2} + a' c e_{1} e_{3} + a c' e_{1} e_{3} + b' c e_{2} e_{3} + b c' e_{2} e_{3} + c c' e_{3}^{2} + a' d e_{1} e_{4} + a d' e_{1} e_{4}
    \]
    \[
      + b' d e_{2} e_{4} + b d' e_{2} e_{4} + c' d e_{3} e_{4} + c d' e_{3} e_{4} + d d' e_{4}^{2}
    \]
    Evaluating the multiplication gets us
    \[
      aa'e_1+a'be_2+ab'e_2 +bb'(-e_1)+a'ce_3+ac'e_3+b'ce_4+bc'e_4+cc'(-e_1)+a'de_4+ad'e_4+b'd(-e_3)+bd'(-e_3)
    \]
    \[
      c'd(-e_2)+cd'(-e_2)+dd'e_1
    \]
    Which we can then simplify to
    \[
      (aa'-bb'-cc'+dd')e_1+(a'b+ab'-c'd-cd')e_2+(a'c+ac'-b'd-bd')e_3+(b'c+bc'+a'd+ad')e_4
    \]
  \item[(b)] First adding
    \[
      \epsilon_1+\epsilon_2=\frac{1}{2}(1\otimes 1+i\otimes i)+\frac{1}{2}(1\otimes 1-i\otimes i)=\frac{2}{2}(1\otimes 1)=1\otimes 1=1
    \]
    Then multiplying
    \[
      \epsilon_1 \epsilon_2=\frac{1}{2}(1\otimes 1+i\otimes i)\frac{1}{2}(1\otimes 1-i\otimes i)=\frac{1}{4}(1\otimes 1+(-1\otimes -1))=\frac{1}{4}(1\otimes 1-1\otimes 1)=0
    \]
    Squaring $\epsilon_1$ gets you
    \[
      \epsilon_1^2=\frac{1}{4}(1\otimes 1+1\otimes 1+i\otimes i +i\otimes i)=\frac{1}{2}(1\otimes 1+i\otimes i)=\epsilon_1
    \]
    Similarly for $\epsilon_2$
    \[
      \epsilon_2^2=\frac{1}{4}(1\otimes 1+1\otimes 1-i\otimes i -i\otimes i)=\frac{1}{2}(1\otimes 1-i\otimes i)=\epsilon_2
    \]
    Then from exercise 7.6.1 from Dummit we have that $A\cong A\epsilon_1\times A\epsilon_2$.
  \item[(c)] Since the space $\bb{C}\times \bb{C}$ can be treated as a vector
    space over $\bb{R}$. This along with the fact that we can pull
    real constants out of the complex conjugate implies that $\varphi$ is
    $\bb{R}$-bilinear.
  \item[(d)] First we calculate $\Phi(\epsilon_1)$ directly
    \[
      \Phi(\epsilon_1)=\Phi(\frac{1}{2}(1\otimes 1+i\otimes i)) =\frac{1}{2}(\Phi(1\otimes 1)+\Phi(i\otimes i)) = \frac{1}{2}((1,1)+(-1,1)) = (0,1)
    \]
    Similarly for $\Phi(\epsilon_2)$ we have
    \[
      \Phi(\epsilon_1)=\Phi(\frac{1}{2}(1\otimes 1-i\otimes i)) =\frac{1}{2}(\Phi(1\otimes 1)-\Phi(i\otimes i)) = \frac{1}{2}((1,1)-(-1,1)) = (1,0)
    \]
    Next we show that $\Phi$ is $\bb{C}$-linear.
    \[
      \Phi(w(z_1\otimes z_2))=\Phi(wz_1\otimes z_2) = (wz_1z_2,wz_1\bar{z_2}) = w(z_1z_2,z_1\bar{z_2})=w\Phi(z_1\otimes z_2)
    \]
    Since $\Phi$ is $\bb{C}$-linear and we have elements that map to $(1,0)$
    and $(0,1)$ it must be surjective as this allows us to reach any
    element using linearity.

    To see that $\Phi$ is injective note that if $\Phi(z_1\otimes z_2)=0$ then either
    $z_1$ or $z_2$ must be zero since $\bb{C}$ is a field. Then
    \[
      z_1\otimes 0 = 0(z_1\otimes 0)= 0\otimes 0 = 0
    \]
    The case for $z_1=0$ follows similarly. As such the kernel of $\Phi$ is trivial
    which implies that $\Phi$ is in fact an isomorphism.
  \end{enumerate}
\end{proof}

%%%%%%%%%%%%%%%%%%%%%%%%%%%%%%%%%%%%%%%%%%%%%%%%%%%%%%%%%%%%%%%%%%%%%%%%%%%%% 
\end{document}
