\documentclass[10pt]{article}
\usepackage[utf8]{inputenc}
\usepackage{amscd}
\usepackage{amsmath}
\usepackage{amssymb}
\usepackage{amsthm}
\usepackage{listings}
\usepackage{enumerate}
\usepackage[all,cmtip]{xy}

\textwidth=15cm \textheight=22cm \topmargin=0.5cm \oddsidemargin=0.5cm \evensidemargin=0.5cm

\newcommand{\sk}{\vskip 10mm}
\newcommand{\bb}[1]{\mathbb{#1}}
\newcommand{\ra}{\rightarrow}

\theoremstyle{plain}
\newtheorem{problem}{Problem}
\newtheorem{lemma}{Lemma}[problem]

\theoremstyle{remark}
\newtheorem{tpart}{}[problem]
\newtheorem*{ppart}{}

\begin{document}

\begin{problem}[7.26]
  Let $M$ and $N$ be finitely generated $R$-modules over a PID $R$. Compute
  $M\otimes_R N$. As a special case, if $M$ is a finite abelian group with
  invariant factors $s_1,\ldots,s_t$ (where as usual we assume that $s_i$
  divides $s_{i+1}$), show that $M\otimes_{\bb{Z}}M$ is a finite group of order
  $\prod_{j=1}^ts_j^{2t-2j+1}$.
\end{problem}

\begin{proof}
  
\end{proof}

\sk

\begin{problem}[7.30]
  \begin{enumerate}
  \item[(a)] Let $F$ be a field and $K$ a field containing $F$.
    If $f(X)\in F[X]$, show that there is an isomorphism of $K$-algebras:
    \[
      K\otimes_F(F[X]/\langle f(X)\rangle\cong K[X]/\langle f(X)\rangle)
    \]
  \item[(b)] By choosing $F, f(x)$, and $K$ appropriately, find an
    example of two fields $K$ and $L$ containing $F$ such that the
    $F$-algebra $K\otimes_F L$ has nilpotent elements.
  \end{enumerate}
\end{problem}

\begin{proof}
  
\end{proof}

\sk


\begin{problem}[7.31]
  Let $F$ be a field. Show that $F[X,Y]\cong F[X]\otimes_F F[Y]$ where the
  isomorphism is an isomorphism of $F$-algebras.
\end{problem}

\begin{proof}
  
\end{proof}

\sk

\begin{problem}[7.34]
  Let $F$ be a field, $V$ and $W$ finite-dimensional vector spaces over $F$,
  and let $T\in\text{End}_F(V),SS\in\text{End}(W)$.
  \begin{enumerate}
  \item[(a)] If $\alpha$ is an eigenvalue of $S$ and $\beta$ is an eigenvalue of
    $T$, show that the product $\alpha\beta$ is an eigenvalue of $S\otimes T$.
  \item[(b)] If $S$ and $T$ are diagonalizable, show that $S\otimes T$
    is diagonalizable.
  \end{enumerate}
\end{problem}

\begin{proof}
  
\end{proof}

\sk

\begin{problem}[10.4.3]
  Show that $\bb{C}\otimes_{\bb{R}}\bb{C}$ and $\bb{C}\otimes_{\bb{C}}\bb{C}$
  are both left $\bb{R}$ modules but are not isomorphic as $\bb{R}$-modules.
\end{problem}

\begin{proof}
  
\end{proof}

\sk

\begin{problem}[10.4.5]
  Let $A$ be a finite abelian group of order $n$ and let $p^k$ be the largest
  power of a prime $p$ dividing $n$. Prove that
  $\bb{Z}/p^k\bb{Z}\otimes_{\bb{Z}}A$ is isomorphic to the Sylow $p$-subgroup
  of $A$.
\end{problem}

\begin{proof}
  
\end{proof}

\sk

\begin{problem}[10.4.10]
  Suppose $R$ is commutative and $N\cong R^n$ is a free $R$-module of rank
  $n$ with $R$-module basis $e_1,\ldots,e_n$.
  \begin{enumerate}
  \item[(a)] For any nonzero $R$-module $M$ show that every element
    of $M\otimes N$ can be written uniquely in the form
    $\sum_{i=1}^n m_i\otimes e_i$ where $m_i\in M$. Deduce that if
    $\sum_{i=1}^nm_i\otimes e_i=0$ in $M\otimes N$ then $m_i=0$ for
    $i=1,\ldots,n$.
  \item[(b)] Show that if $\sum m_i\otimes n_i=0$ in $M\otimes N$ where the
    $n_i$ are merely assumed to be $R$-linearly independent then it is not
    necessarily true that all the $m_i$ are 0. [Consider
    $R=\bb{Z},n=1,M=\bb{Z}/2\bb{Z}$, and the element $1\otimes 2$.]
  \end{enumerate}
\end{problem}

\begin{proof}
  
\end{proof}

\sk

\begin{problem}[10.4.24]
  Prove that the extension of scalars from $\bb{Z}$ to the Gaussian integers
  $\bb{Z}[i]$ of the ring $\bb{R}$ is isomorphic to $\bb{C}$ as a ring:
  $\bb{Z}[i]\otimes_{\bb{Z}}\bb{R}\cong\bb{C}$ as rings.
\end{problem}

\begin{proof}
  
\end{proof}

\sk

\begin{problem}[10.4.27]
  \begin{enumerate}
  \item[(a)] Write down a formula for the multiplication of two elements
    $a\cdot 1+b\cdot e_2+c\cdot e_3+d\cdot e_4$ and
    $a'\cdot 1+b'\cdot e_2+c'\cdot e_3+d'\cdot e_4$ in the example
    $A=\bb{C}\otimes_{\bb{R}}\bb{C}$ following proposition 21
    (where $1=1\otimes 1$ is the identity of $A$).
  \item[(b)] Let $\epsilon_1=\frac{1}{2}(1\otimes 1+i\otimes i)$ and
    $\epsilon_2=\frac{1}{2}(1\otimes 1-i\otimes i)$. Show that
    $\epsilon_1\epsilon_2=0,\epsilon_1+\epsilon_2=1$, and
    $\epsilon_j^2=\epsilon_j$ for $j=1,2$ ($\epsilon_1$ and $\epsilon_2$ are
    called orthogonal idempotents in $A$). Deduce that $A$ is isomorphic
    as a ring to the direct product of two principle ideals:
    $A\cong A\epsilon_1\times A\epsilon_2$ (cf. Exercise 1, Section 7.6).
  \item[(c)] Prove that the map
    $\varphi:\bb{C}\otimes\bb{C}\rightarrow\bb{C}\times\bb{C}$ by
    $\varphi(z_1,z_2)=(z_1z_2,z_1\bar{z_2})$, where $\bar{z_2}$
    denotes the complex conjugate of $z_2$, is an $\bb{R}$-bilinear map.
  \item[(d)] Let $\Phi$ be the $\bb{R}$-module homomorphism from $A$ to
    $\bb{C}\times\bb{C}$ obtained from $\varphi$ in (c). Show that
    $\Phi(\epsilon_1)=(0,1)$ and $\Phi(\epsilon_2)=(1,0)$. Show also that $\Phi$
    is $\bb{C}$-linear, where the action of $\bb{C}$ is on the left tensor
    factor in A and on both factors in $\bb{C}\times\bb{C}$. Deduce that
    $\Phi$ is surjective. Show that $\Phi$ is a $\bb{C}$-algebra isomorphism.
  \end{enumerate}
\end{problem}

\begin{proof}
  
\end{proof}

%%%%%%%%%%%%%%%%%%%%%%%%%%%%%%%%%%%%%%%%%%%%%%%%%%%%%%%%%%%%%%%%%%%%%%%%%%%%%
\end{document}
