\documentclass[10pt]{article}
\usepackage[utf8]{inputenc}
\usepackage{amscd}
\usepackage{amsmath}
\usepackage{amssymb}
\usepackage{amsthm}
\usepackage{listings}
\usepackage{enumerate}

\textwidth=15cm \textheight=22cm \topmargin=0.5cm \oddsidemargin=0.5cm \evensidemargin=0.5cm

\newcommand{\sk}{\vskip 10mm}
\newcommand{\bb}[1]{\mathbb{#1}}
\newcommand{\ra}{\rightarrow}

\theoremstyle{plain}
\newtheorem{problem}{Problem}
\newtheorem{lemma}{Lemma}[problem]

\theoremstyle{remark}
\newtheorem{tpart}{}[problem]
\newtheorem*{ppart}{}

\begin{document}

\begin{problem}[13.6.6]
  Prove that for $n$ odd, $n>1$, $\Phi_{2n}(x)=\Phi_n(-x)$.
\end{problem}

\begin{proof}
  
\end{proof}

\sk

\begin{problem}[13.6.9]
  Suppose $A$ is an $n\times n$ matrix over $\bb{C}$ for which $A^k=I$
  for some integer $k\geq1$. Show that $A$ can be diagonalized.
  Show that the matrix
  $A=\left(\begin{array}{cc}1&\alpha\\0,1\\ \end{array}\right)$
  where $\alpha$ is an element of a field of characteristic $p$ satisfies
  $A^p=I$ and cannot be diagonalized if $\alpha\neq 0$.
\end{problem}

\begin{proof}
  
\end{proof}

\sk

\begin{problem}[13.6.10]
  Let $\varphi$ denote the Frobenius map $x\mapsto x^p$ on the finite field
  $\bb{F}_{p^n}$. Prove that $\phi$ gives an isomorphism of $\bb{F}_{p^n}$
  to itself. Prove that $\varphi^n$ is the identity map and that no
  lower power of $\varphi$ is the identity.
\end{problem}

\begin{proof}
  
\end{proof}

\sk

\begin{problem}[13.6.13]
  This exercise outlines a proof of Wedderburn's Theorem that a
  finite division ring $D$ is a field.
  \begin{enumerate}
  \item[(a)] Let $Z$ denote the center of $D$. Prove that $Z$ is
    a field containing $\bb{F}_p$ for some prime $p$. If $Z=\bb{F}_q$
    prove that $D$ has order $q^n$ for some integer $n$.
    [$D$ is a vector space over $Z$].
    
  \item[(b)] The nonzero elements $D^\times$ of $D$ form a multiplicative group.
    For any $x\in D^\times$ show that the nonzero elements of $D$ which commute
    with $x$ form a division ring which contains $Z$. Show that this
    division ring is of order $q^m$ for some integer $m$ and that $m<n$
    if $x$ is not an element of $Z$.
    
  \item[(c)] Show that the class equation for the group $D^\times$ is
    \[
      q^n-1=(q-1)+\sum_{i=1}^r\frac{q^n-1}{|C_{D^\times}(x_i)|}
    \]
    where $x_1,x_2,\ldots,x_r$ are representatives of the distinct
    conjugacy classes in $D^\times$ not contained in the center of $D^\times$.
    Conclude from (b) that for each $i$, $|C_{D^\times}(x_i)|=q^{m_i}-1$
    for some $m_i<n$.
    
  \item[(d)] Prove that since $\frac{q^n-1}{q^{m_i}-1}$ is an integer
    (namely, the index $|D^\times:C_{D^\times}(x_i)|$) then $m_i$ divides $n$.
    Conclude that $\Phi_n(x)$ divides $(x^n-1)/(x^{m_i}-1)$ and hence that the
    integer $\Phi_n(q)$ divides $(q^n-1)/(q^{m_i}-1)$ for $i=1,2,\ldots,r$.
    
  \item[(e)] Prove that (c) and (d)e imply that
    $\Phi(q)=\prod_{\zeta\ \mathrm{primitive}}(q-\zeta)$ divides $q-1$.
    Prove that $|q-\zeta|>q-1$ (complex absolute value) for any root
    of unity $\zeta\neq 1$ [note that 1 is the closest point on the unit circle
    in $\bb{C}$] to the point $q$ on the real line]. Conclude that
    $n=1$, i.e., that $D=Z$ is a field.
  \end{enumerate}
\end{problem}

\begin{proof}
  
\end{proof}

\sk

\begin{problem}[14.1.4]
  Prove that $\bb{Q}[\sqrt{2}]$ and $\bb{Q}[\sqrt{3}]$ are not isomorphic.
\end{problem}

\begin{proof}
  
\end{proof}

\sk

\begin{problem}[14.2.4]
  Let $p$ be a prime. Determine the elements of the Galois group of $x^p-2$.
\end{problem}

\begin{proof}
  
\end{proof}

\sk

\begin{problem}[14.2.5]
  Prove that the Galois group of $x^p-2$ for $p$ a prime is isomorphic to
  the group of matrices $\left(\begin{array}{cc}a&b\\0&1\\ \end{array}\right)$
  where $a,b\in\bb{F}_p,a\neq 0$.
\end{problem}

\begin{proof}
  
\end{proof}

\sk

\begin{problem}[14.2.14]
  Show that $\bb{Q}(\sqrt{2+\sqrt{2}})$ is a cyclic quartic field,
  i.e., is a Galois extension of degree 4 with cyclic Galois group.
\end{problem}

\begin{proof}
  
\end{proof}

\sk

%%%%%%%%%%%%%%%%%%%%%%%%%%%%%%%%%%%%%%%%%%%%%%%%%%%%%%%%%%%%%%%%%%%%%%%%%%%%%
\end{document}
