\documentclass[10pt]{article}
\usepackage[utf8]{inputenc}
\usepackage{amscd}
\usepackage{amsmath}
\usepackage{amssymb}
\usepackage{amsthm}
\usepackage{listings}
\usepackage{enumerate}

\textwidth=15cm \textheight=22cm \topmargin=0.5cm \oddsidemargin=0.5cm \evensidemargin=0.5cm

\newcommand{\sk}{\vskip 10mm}
\newcommand{\bb}[1]{\mathbb{#1}}
\newcommand{\ra}{\rightarrow}

\theoremstyle{remark}
\newtheorem{problem}{Problem}
\newtheorem{lemma}{Lemma}[problem]

\theoremstyle{remark}
\newtheorem{tpart}{}[problem]
\newtheorem*{ppart}{}

\begin{document}

\begin{problem}[13.6.6]
  Prove that for $n$ odd, $n>1$, $\Phi_{2n}(x)=\Phi_n(-x)$.
\end{problem}

\begin{proof}
  Begin with $\Phi_n(-x)$. Then we have that
  \[
    \Phi_n(-x) = \prod_{1\leq d<n|(d,n)=1}(-x-\zeta_n^d)
  \]
  If we pull out the negatives we get
  \[
    \Phi_n(-x) = (-1)^{\varphi(n)}\prod_{1\leq d<n|(d,n)=1}(x-\zeta_n^{d+n/2})
  \]
  Since $\varphi(m)$ is even for $m\geq 3$ we can safely remove it. Then
  we change the base of $\zeta_n$ to get
  \[
    \Phi_n(-x) = (-1)^{\varphi(n)}\prod_{1\leq d<n|(d,n)=1}(x-\zeta_{2n}^{2d+n})
  \]
  All of the $2d+n$ are greater than or equal to 1 and less than
  $2n$. Moreover as $n$ is odd, greater than 1, and $\gcd(d,n)=1$
  we have that $\gcd(2d,n)=1$. Since $\deg \Phi_{2n}(x)=\varphi(2n)=\varphi(n)$ and
  there are $\varphi(n)$ factors in the above product we must have all of
  the factors for $\Phi_{2n}(x)$.
  
  Therefore
  \[
    \Phi_n(-x)=\Phi_{2n}(x)
  \]

  for $n$ odd and $n>1$.
\end{proof}

\sk

\begin{problem}[13.6.9]
  Suppose $A$ is an $n\times n$ matrix over $\bb{C}$ for which $A^k=I$
  for some integer $k\geq1$. Show that $A$ can be diagonalized.
  Show that the matrix
  $A=\left(\begin{array}{cc}1&\alpha\\0&1\\ \end{array}\right)$
  where $\alpha$ is an element of a field of characteristic $p$ satisfies
  $A^p=I$ and cannot be diagonalized if $\alpha\neq 0$.
\end{problem}

\begin{proof}
  Let $J$ be the Jordan normal form of $A$. This will exist since
  we are working over the complex numbers. If $J$ is diagonalizable
  then $A$ will be as well. However because we have the relation
  $A^k-I_n=0$ for some $k>1$ it follows that the characteristic polynomial
  of $A$ will be $x^k-1$. However this has all distinct roots. As such
  the block matrices in $J$ will have to be $1\times 1$ since the eigenvalues
  are distinct. Thus $J$ is a diagonal matrix and so is $A$.

  \sk
  
  For the second part first note that
  \[
    \left(
      \begin{array}{cc}
        1&\alpha\\
        0&1\\
      \end{array}
    \right)^k
    =
    \left(
      \begin{array}{cc}
        1&k\alpha\\
        0&1\\
      \end{array}
    \right)
  \]
  which demonstrates that $A^p=I$ since we are in a field of characteristic
  $p$. If we calculate the characteristic polynomial of $A$ where $\alpha\neq 0$
  we get $(x-1)^2$. Since the eigenvalues are not unique it will not be
  diagonalizable.
\end{proof}

\sk

\begin{problem}[13.6.10]
  Let $\varphi$ denote the Frobenius map $x\mapsto x^p$ on the finite field
  $\bb{F}_{p^n}$. Prove that $\phi$ gives an isomorphism of $\bb{F}_{p^n}$
  to itself. Prove that $\varphi^n$ is the identity map and that no
  lower power of $\varphi$ is the identity.
\end{problem}

\begin{proof}
  Since powers distribute over multiplication it is clear that
  $\varphi$ preserves multiplication. The fact that it preserves addition
  follows from $\bb{F}_{p^n}$ being of characteristic $p$ as:
  \[(x+y)^p=\sum_0^p{p\choose k}x^ky^{p-k}=x^p+y^p \]

  Now we must show that the map is both injective and surjective.
  We will start with injectivity. Suppose that $x^p=1$. Then
  \[x^p-1^p=(x-1)^p=0 \]
  Which implies that $x=1$ since we are in a field. Since the
  kernel is trivial it follows that $\varphi$ is injective.

  For surjectivity note that $F_{p^n}^*$ is a multiplicative
  group of order $p^n-1$. As such given $y\in F_{p^n}^*$ we have
  that $y^{p^n}=y$. It then follows that
  \[\left(y^{p^{n-1}}\right)^p=\varphi\left(y^{p^{n-1}}\right)=y\]
  which demonstrates that $\varphi$ is surjective.

  Therefore the Frobenius map $\varphi$ is an isomorphism.

  \sk

  For the latter portion note that $\varphi^n(x)=x^{p^n}$ which
  is equal to $x$ from the argument made earlier. However
  this cannot occur from $m<n$. If it did then we would have
  that $x^{p^m-1}=x$ for all $x\in\bb{F}_{p^n}$. This would imply that
  the orders of all elements in $\bb{F}_{p^n}$ is at most $p^m-1$.
  However this is a contradiction as the multiplicative groups for
  finite fields are cyclic.
\end{proof}

\sk

\begin{problem}[13.6.13]
  This exercise outlines a proof of Wedderburn's Theorem that a
  finite division ring $D$ is a field.
  \begin{enumerate}
  \item[(a)] Let $Z$ denote the center of $D$. Prove that $Z$ is
    a field containing $\bb{F}_p$ for some prime $p$. If $Z=\bb{F}_q$
    prove that $D$ has order $q^n$ for some integer $n$.
    [$D$ is a vector space over $Z$].
    
  \item[(b)] The nonzero elements $D^\times$ of $D$ form a multiplicative group.
    For any $x\in D^\times$ show that the nonzero elements of $D$ which commute
    with $x$ form a division ring which contains $Z$. Show that this
    division ring is of order $q^m$ for some integer $m$ and that $m<n$
    if $x$ is not an element of $Z$.
    
  \item[(c)] Show that the class equation for the group $D^\times$ is
    \[
      q^n-1=(q-1)+\sum_{i=1}^r\frac{q^n-1}{|C_{D^\times}(x_i)|}
    \]
    where $x_1,x_2,\ldots,x_r$ are representatives of the distinct
    conjugacy classes in $D^\times$ not contained in the center of $D^\times$.
    Conclude from (b) that for each $i$, $|C_{D^\times}(x_i)|=q^{m_i}-1$
    for some $m_i<n$.
    
  \item[(d)] Prove that since $\frac{q^n-1}{q^{m_i}-1}$ is an integer
    (namely, the index $|D^\times:C_{D^\times}(x_i)|$) then $m_i$ divides $n$.
    Conclude that $\Phi_n(x)$ divides $(x^n-1)/(x^{m_i}-1)$ and hence that the
    integer $\Phi_n(q)$ divides $(q^n-1)/(q^{m_i}-1)$ for $i=1,2,\ldots,r$.
    
  \item[(e)] Prove that (c) and (d)e imply that
    $\Phi(q)=\prod_{\zeta\ \mathrm{primitive}}(q-\zeta)$ divides $q-1$.
    Prove that $|q-\zeta|>q-1$ (complex absolute value) for any root
    of unity $\zeta\neq 1$ [note that 1 is the closest point on the unit circle
    in $\bb{C}$] to the point $q$ on the real line]. Conclude that
    $n=1$, i.e., that $D=Z$ is a field.
  \end{enumerate}
\end{problem}

\begin{proof}
  
\end{proof}

\sk

\begin{problem}[14.1.4]
  Prove that $\bb{Q}[\sqrt{2}]$ and $\bb{Q}[\sqrt{3}]$ are not isomorphic.
\end{problem}

\begin{proof}
  Suppose that $\bb{Q}[\sqrt{2}]$ and $\bb{Q}[\sqrt{3}]$ were isomorphic.
  Then there would be an isomorphism $\varphi:\bb{Q}[\sqrt{2}]\rightarrow\bb{Q}[\sqrt{3}]$.
  Let $\varphi(\sqrt{2})=a+b\sqrt{3}$. Then we have that
  \[ \varphi(2)=\varphi(1+1)=\varphi(1)+\varphi(1)=2\]
  it then follows that $(a+b\sqrt{3})^2=2$. However by expanding we get
  \[ a^2+3b^2+2ab\sqrt{3}=2 \]
  which implies that either $a$ or $b$ is zero since we are in a field.
  If $b=0$ then $a^2=2$ which implies that $\sqrt{2}\in\bb{Q}[\sqrt{3}]$ which is
  a contradiction. On the other hand if $a=0$ then $b^2=2/3$ which
  implies that $\sqrt{3}b=\sqrt{2}$. Then $\sqrt{2/3}\in\bb{Q}[\sqrt{3}]$
  once again which is a contradiction.

  Therefore the fields $\bb{Q}[\sqrt{2}]$ and $\bb{Q}[\sqrt{3}]$ are
  not isomorphic.
\end{proof}

\sk

\begin{problem}[14.2.4]
  Let $p$ be a prime. Determine the elements of the Galois group of $x^p-2$.
\end{problem}

\begin{proof}
  
\end{proof}

\sk

\begin{problem}[14.2.5]
  Prove that the Galois group of $x^p-2$ for $p$ a prime is isomorphic to
  the group of matrices $\left(\begin{array}{cc}a&b\\0&1\\ \end{array}\right)$
  where $a,b\in\bb{F}_p,a\neq 0$.
\end{problem}

\begin{proof}
  
\end{proof}

\sk

\begin{problem}[14.2.14]
  Show that $\bb{Q}(\sqrt{2+\sqrt{2}})$ is a cyclic quartic field,
  i.e., is a Galois extension of degree 4 with cyclic Galois group.
\end{problem}

\begin{proof}
  For the sake of brevity let $\alpha:=\sqrt{2+\sqrt{2}}$.
  We know from a prior homework that the degree of $\bb{Q}(\alpha)$
  is $4$. We also know that minimal polynomial for $\alpha$ is
  $x^4-4x^2+2$ whose roots are $\pm\sqrt{2\pm\sqrt{2}}$. Thus the Galois group
  for this field must be of size 4. Define $\sigma:\bb{Q}(\alpha)\rightarrow\bb{Q}(\alpha)$ by
  its action on the roots
  \[\alpha\mapsto\alpha^3-\alpha,\quad-\alpha\mapsto-\alpha^3+\alpha,\quad\alpha^3-\alpha\mapsto-\alpha,\quad-\alpha^3-\alpha\mapsto\alpha \]

  \textbf{Probably want to justify why this is a homomorphism at all}
  
  This is of order 4. As such the Galois group must be isomorphic to
  $\bb{Z}_4$.
\end{proof}

\sk

%%%%%%%%%%%%%%%%%%%%%%%%%%%%%%%%%%%%%%%%%%%%%%%%%%%%%%%%%%%%%%%%%%%%%%%%%%%%%
\end{document}
