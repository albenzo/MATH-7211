\documentclass[10pt]{article}
\usepackage[utf8]{inputenc}
\usepackage{amscd}
\usepackage{amsmath}
\usepackage{amssymb}
\usepackage{amsthm}
\usepackage{listings}
\usepackage{enumerate}
\usepackage[all,cmtip]{xy}

\textwidth=15cm \textheight=22cm \topmargin=0.5cm \oddsidemargin=0.5cm \evensidemargin=0.5cm

\newcommand{\sk}{\vskip 10mm}
\newcommand{\bb}[1]{\mathbb{#1}}
\newcommand{\ra}{\rightarrow}
\newcommand{\id}{\mathrm{id}}

\theoremstyle{plain}
\newtheorem{problem}{Problem}
\newtheorem{lemma}{Lemma}[problem]

\theoremstyle{remark}
\newtheorem{tpart}{}[problem]
\newtheorem*{ppart}{}

\begin{document}

\begin{problem}
  Prove that the following conditions on an $R$-module $P$ are equivalent.
  \begin{enumerate}
  \item[(a)] $P$ is projective.
  \item[(b)] $P$ is isomorphic to a direct summand of a free $R$-module.
  \item[(c)] If $f:M\rightarrow P$ is surjective, then there exists and
    $R$-module homomorphism $g:P\rightarrow M$ such that $f\circ g=\id_P$.
  \end{enumerate}
\end{problem}

\begin{proof}
  First we will show that (a) implies (c).
  Let $P$ be a projective module and $f:M\rightarrow P$ be surjective. Then
  the identity map $\id_P$ fulfills the projective conditions and
  there exists a map $g:P\rightarrow M$ such that $f\circ g = \id_P$.

  Next we will show that (c) implies (b). Let $P$ be a module
  that fulfills property (c). Then let $F$ be the free group
  with generators the elements of $P$ and $Q$ the free group
  with generators the relations of $P$. This creates a short
  exact sequence inclusion and projection of the form
  \[
    \xymatrix{
      0 \ar[r] & Q \ar[r]^i & F \ar[r]^\pi & P \ar[r] & 0
    }
  \]
  However condition (c) states that this sequence splits.
  As such $F\cong Q\oplus P$ by the splitting lemma which implies that
  condition (b) holds.

  Finally we show that (b) implies (a). Let $F=P\oplus Q$ where $F$
  is free and let $\pi:F\rightarrow P$ be the projection map. Then let
  $f:P\rightarrow N$ and $g:M\rightarrow N$ where $g$ is surjective.

  Let $x$ be a generator of $F$ and $n_x$ as $f\circ\pi(x)\in N$. By surjectivity
  of $g$ there exists $m_x\in M$ such that $g(m_x)=n_x$. By the universal
  property of free modules there exists a unique map $\bar{f}:F\rightarrow M$ such
  that $g\circ\bar{f}(x)=f\circ\pi(x)$. Now define $\tilde{f}:P\rightarrow M$ by $\tilde{f}(p)=\bar{f}(p+0_q)$.
  Then
  \[
    g\circ\tilde{f}(p)=f\circ\pi(p+0_q)=f(p)
  \]
  which completes the proof.
\end{proof}

\sk

\begin{problem}
  Let $F$ be a field and let $R=F\times F$. Let $e=(1,0)\in R$ and let  $P=Re$.
  Show that $P$ is a projective $R$-module, but that $P$ is not a
  free $R$-module.
\end{problem}

Note that $R$ is a free module over itself. Since $P\oplus R(0,1)\cong R$
we have that $P$ is projective by problem 1. However $P$ is
not free as if we let $x,y\in F\setminus\{0\}$ then $(0,y)\cdot(x,0)=0$ even
though neither element is zero.

\sk

\begin{problem}
  Show that if $R$ is a semisimple ring, then so is $M_n(R)$.
\end{problem}

\begin{proof}
  By Theorem $7.1.28$ from Adkins' book a ring is semisimple if and
  only if every $R$-module is semisimple. Since $M_n(R)$ is an $R$-module
  it is semisimple.
\end{proof}

\sk

\begin{problem}
  Show that if $R$ is a semisimple ring and $I$ is any ideal, then $R/I$
  is also semisimple.
\end{problem}

\begin{proof}
  Suppose that $R=\bigoplus_\alpha M_\alpha$ is a semisimple ring and let $\pi:R\rightarrow R/I$ be the
  projection map. Then $\pi(M_\alpha)$ is an ideal of $R/I$ which implies that
  $R/I\cong \bigoplus_\alpha \pi(M_\alpha)$. Moreover since the preimages of ideals are ideals
  the components $\pi(M_\alpha)$ must also be simple otherwise it would violate
  the simplicity of the $M_\alpha$s.

  Therefore since the $\pi(M_\alpha)$s are simple and ideals of $R/I$ we have
  a decomposition of $R/I$ into simple submodules as a $R/I$ module.
  It then follows that $R/I$ is a semisimple ring.
\end{proof}

\sk

\begin{problem}[7.2]
  Let $F$ be a field and let
  \[
    R =
    \left\{
      \left[
        \begin{array}{cc}
          a&b\\
          0&c\\
        \end{array}
      \right]
      |
      a,b,c\in F
    \right\}
  \]
  be the ring of upper triangular matrices over $F$. Let $M=F^2$ and make
  $M$ into a (left) $R$-module by matrix multiplication. Show that
  $\text{End}_R(M)\cong F$. Conclude that the converse of Schur's lemma
  is false, i.e., $\text{End}_R(M)$ can be a division ring without
  $M$ being a simple $R$-module.
\end{problem}

\begin{proof}
  Define a map $\psi:F\rightarrow End_R(M)$ by $\Psi(x)=\varphi_x$ where $\varphi_x$
  is defined as
  \[
    \varphi_x\left(
      \begin{array}{c}
        d\\
        e
      \end{array}
    \right)
    =
    \left(
      \begin{array}{c}
        xd\\
        xe\\
      \end{array}
    \right)
  \]
  Since the maps $\varphi_x$ are equivalent to scalar multiplication by
  field elements it is clear from the definition that $\varphi_x$ are in
  fact endomorphisms and respect the module structure. Now we
  will show that $\psi$ is indeed an isomorphism.

  The fact that the kernel of $\psi$ is trivial follows from the fact
  that we are multiplying by field elements and as such there are
  no zero divisors. However to show that $\psi$ is surjective we
  must utilize the fact that the endomorphisms play nice with
  the module structure.

  First note that for $\phi\in End_R(M)$ we have
  \[
    \left(
      \begin{array}{cc}
        a&b\\
        0&c\\
      \end{array}
    \right)
    \phi
    \left(
      \begin{array}{c}
        d\\
        e\\
      \end{array}
    \right)
    =
    \phi
    \left(
      \left(
        \begin{array}{cc}
          a&b\\
          0&c\\
        \end{array}
      \right)
      \left(
        \begin{array}{c}
          d\\
          e\\
        \end{array}
      \right)
    \right)
  \]
  If we split $\phi$ into two pieces like so
  \[
    \phi\left(
      \begin{array}{c}
        d\\
        e\\
      \end{array}
    \right)
    =
    \left(
      \begin{array}{c}
        \phi_1(d,e)\\
        \phi_2(d,e)\\
      \end{array}
    \right)
  \]
  The module structure gives us two equations
  \[
    a\phi_1(d,e)+b\phi_2(d,e)=\phi_1(ad+be,ce)
  \]
  and
  \[
    c\phi_2(d,e)=\phi_2(ad+be,ce)
  \]
  Note that we could define a new endomorphism by swapping $\phi_1$ and
  $\phi_2$. As such any statement we make about one applies to the other.

  First if we let $a=1,b=c=0$ in the latter equation this gives us that
  $\phi_2(d,0)=0$. Thus $\phi_1$ and $\phi_2$ are zero whenever the right coordinate
  are zero. Next let $a=1,b=-1,c=0$ in the first equation and we get that
  $\phi_1(d,e)=\phi_2(d,e)$. Finally if we set $b=0,c=1$ and let $a\in F$ we
  get that we can pull constants out of the right term. Similarly
  setting $a=1,b=0$ and letting $c\in F$ we get that we can pull constants
  out of the right term. This implies that $\phi_1$ and $\phi_2$ are equivalent to
  multiplying by $\phi(1,1)=x$. Thus the map $\phi=\varphi_x$ and implying that $\psi$ is
  surjective.

  Therefore $\psi$ is in fact an isomorphism and as such $F\cong End_R(M)$.
  Since $M$ is not simple but it's endomorphisms form a field this
  is a counterexample to the converse of Schur's lemma.
\end{proof}

\sk

\begin{problem}[7.4]
  An $R$-module $M$ is said to satisfy the descending chain condition (DCC)
  on submodules if any strictly decreasing chain of submodules of $M$ of
  finite length.
  \begin{enumerate}
  \item[(a)] Show that if $M$ satisfies the DCC, then any nonempty set
    of submodules of $M$ contains a minimal element.
  \item[(b)] Show that $\ell(M)<\infty$ if and only if $M$ satisfies $M$
    satisfies both the ACC and DCC.
  \end{enumerate}
\end{problem}

\begin{proof}
  \begin{enumerate}
  \item[(a)] Let $\mathcal{M}$ be the set of submodules of $M$ partially
    ordered by reverse inclusion. Since $M$ satisfies DCC every chain
    in $\mathcal{M}$ will have a maximal element and as such
    by Zorn's lemma there is a maximal element of $\mathcal{M}$ that
    corresponds to a minimal submodule.
  \item[(b)] Suppose that $\ell(M)<\infty$. Then given a series of submodules
    $\{N\}$ we can refine it to a composition series $\{N'\}$ which
    will be of finite length. Since the length of $\{N\}$ is less
    than that of $\{N'\}$ it must also have finite length and as
    such has a minimal point and a maximal point. Since
    this holds for any series it follows that $M$ satisfies
    both ACC and DCC.

    Otherwise suppose that $\ell(M)=\infty$ and let $\{ N\}$ be a composition
    series. Since all composition series have the same length
    this series must be infinite and as such $M$ breaks either
    ACC or DCC.
  \end{enumerate}
\end{proof}

\sk

\begin{problem}[7.5]
  Let
  \[
    R=
    \left\{
      \left[
        \begin{array}{cc}
          a&b\\
          0&c\\
        \end{array}
      \right]
      :
      a,b\in\bb{R}; c\in\bb{Q}
    \right\}
  \]
  $R$ is a ring under matrix addition and multiplication. Show that $R$
  satisfies the ACC and DCC on left ideals but neither chain condition
  is valid for right ideals. Thus $R$ is of finite length as a left
  $R$-module, but $\ell(R)=\infty$ as a right $R$-module.
\end{problem}

Starting with $R$ as a left $R$-module we can construct a composition
series as
\[
  \left[
    \begin{array}{cc}
      0&b\\
      0&0\\
    \end{array}
  \right]
  \subset
  \left[
    \begin{array}{cc}
      a&b\\
      0&0\\
    \end{array}
  \right]
  \subset
  \left[
    \begin{array}{cc}
      a&b\\
      0&c\\
    \end{array}
  \right]
\]
the quotients of which are isomorphic to $\bb{R},\bb{R},$ and $\bb{Q}$
respectively. Since we have a composition series of finite length
it must be that $\ell(R)<\infty$ as a left module and from the prior problem
satisfies ACC and DCC.

However for $R$ as a right module start with the submodule that
has $a=c=0$ and $b\in \bb{Q}$. Multiplication looks like
\[
  \left[
    \begin{array}{cc}
      0&b\\
      0&0\\
    \end{array}
  \right]
  \left[
    \begin{array}{cc}
      d&e\\
      0&f\\
    \end{array}
  \right]
  =
  \left[
    \begin{array}{cc}
      0&cf\\
      0&0\\
    \end{array}
  \right]
\]

Since $f\in\bb{Q}$ it will be closed under the action of $R$.
Then create an ascending series of submodules by
adjoining roots of primes $\bb{Q}[\sqrt{2},\sqrt{3},\ldots]$.
Since there are infinitely many primes this chain is infinite
and as such $\ell(R)=\infty$ as a right $R$-module.

For the descending series let $\mathcal{B}$ be a basis for $\bb{R}$ over
$\bb{Q}$. Then do the same construction as above but instead each
time take away an element of $\mathcal{B}$. This will create an
infinite descending series.

\sk

\begin{problem}[7.11]
  Let $F$ be a field, let $V$ be a finite-dimensional vector space over $F$,
  and let $T\in \text{End}_F(V)$. We shall say that $T$ is semisimple if the
  $F[X]$-module $V_T$ is semisimple. If $A\in M_n(F)$, we shall say that $A$
  is semisimple if the linear transformation $T_A: F^n\rightarrow F^n$
  (multiplication by $A$) is semisimple. Let $\bb{F}_2$ be the field with
  $2$ elements and let $F=\bb{F}_2(Y)$ be the rational function field in
  the indeterminate $Y$, and let $K=F[X]/\langle X^2+Y\rangle$. Since
  $X^2+Y\in F[X]$ is irreducible, $K$ is a field containing
  $F$ as a subfield. Now let
  \[
    A=C(X^2+Y)=
    \left[
      \begin{array}{cc}
        0&Y\\
        1&0\\
      \end{array}
    \right]
    \in M_2(F)
  \]
  Show that $A$ is semisimple when considered in $M_2(F)$ but $A$ is not
  semisimple when considered in $M_2(K)$. Thus, semisimplicity of a matrix
  is not necessarily preserved when one passes to a larger field. However,
  prove that if $L$ is a subfield of the complex numbers $\bb{C}$, then
  $A\in M_n(L)$ is semisimple if and only if it is also semisimple as a complex
  matrix.
\end{problem}

From Adkins' book the linear transformation will be semisimple as
described above if the minimal polynomial is the product of irreducible
factors. The minimal polynomial for $A$ is $X^2+Y$ which is irreducible
for $F$ but not irreducible for $K$.

\begin{proof}
  Let $\bb{L}$ be a subfield of $\bb{C}$ and $A\in M_n(\bb{L})$. Then as
  above the matrix $A$ is semisimple only when its minimal polynomial
  is the product of distinct irreducible factors. If $A$ is irreducible as
  a complex matrix then it will definitely be irreducible as a matrix
  over $\bb{L}$. On the other hand if the minimal polynomial of $A$
  is the product of distinct irreducible factors over $\bb{L}$ then it
  will factor into linear terms when we move to $\bb{C}$. However none
  of these terms will be shared between factors in $\bb{L}$ as otherwise
  it would not be a proper minimal polynomial.

  Therefore a matrix is semisimple on a subfield of the complex numbers
  if and only if it is semisimple over the complex numbers themselves.
\end{proof}


\sk

\begin{problem}[7.17]
  \begin{enumerate}
  \item[(a)] Prove that if $R$ is a semisimple ring and $I$ is an ideal,
    then $R/I$ is semisimple.
  \item[(b)] Show (by example) that a subring of a semisimple ring need
    not be semisimple.
  \end{enumerate}
\end{problem}

\begin{proof}
  \begin{enumerate}
  \item[(a)] See problem 4.
  \item[(b)] The rationals are simple, and as such semisimple, since they are
    a field. However the integers are a subring of the rationals and
    the integers are not semisimple.
  \end{enumerate}
\end{proof}

%%%%%%%%%%%%%%%%%%%%%%%%%%%%%%%%%%%%%%%%%%%%%%%%%%%%%%%%%%%%%%%%%%%%%%%%%%%%% 
\end{document}
