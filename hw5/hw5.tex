\documentclass[10pt]{article}
\usepackage[utf8]{inputenc}
\usepackage{amscd}
\usepackage{amsmath}
\usepackage{amssymb}
\usepackage{amsthm}
\usepackage{listings}
\usepackage{enumerate}
\usepackage[all,cmtip]{xy}

\textwidth=15cm \textheight=22cm \topmargin=0.5cm \oddsidemargin=0.5cm \evensidemargin=0.5cm

\newcommand{\sk}{\vskip 10mm}
\newcommand{\bb}[1]{\mathbb{#1}}
\newcommand{\ra}{\rightarrow}
\newcommand{\id}{\mathrm{id}}

\theoremstyle{plain}
\newtheorem{problem}{Problem}
\newtheorem{lemma}{Lemma}[problem]

\theoremstyle{remark}
\newtheorem{tpart}{}[problem]
\newtheorem*{ppart}{}

\begin{document}

\begin{problem}
  Prove that the following conditions on an $R$-module $P$ are equivalent.
  \begin{enumerate}
  \item[(a)] $P$ is projective.
  \item[(b)] $P$ is isomorphic to a direct summand of a free $R$-module.
  \item[(c)] If $f:M\rightarrow P$ is surjective, then there exists and
    $R$-module homomorphism $g:P\rightarrow M$ such that $f\circ g=\id_P$.
  \end{enumerate}
\end{problem}

\begin{proof}
  
\end{proof}

\sk

\begin{problem}
  Let $F$ be a field and let $R=F\times F$. Let $e=(1,0)\in R$ and let  $P=Re$.
  Show that $P$ is a projective $R$-module, but that $P$ is not a
  free $R$-module.
\end{problem}

\begin{proof}
  
\end{proof}

\sk

\begin{problem}
  Show that if $R$ is a semisimple ring, then so is $M_n(R)$.
\end{problem}

\begin{proof}
  
\end{proof}

\sk

\begin{problem}
  Show that if $R$ is a semisimple ring and $I$ is any ideal, then $R/I$
  is also semisimple.
\end{problem}

\begin{proof}
  
\end{proof}

\sk

\begin{problem}[7.2]
  Let $F$ be a field and let
  \[
    R =
    \{
    \left[
      \begin{array}{cc}
        a&b\\
        0&c\\
      \end{array}
    \right]
    |
    a,b,c\in F
    \}
  \]
  be the ring of upper triangular matrices over $F$. Let $M=F^2$ and make
$M$ into a (left) $R$-module by matrix multiplication. Show that
  $\text{End}_R(M)\cong F$. Conclude that the converse of Schur's lemma
  is false, i.e., $\text{End}_R(M)$ can be a division ring without
  $M$ being a simple $R$-module.
\end{problem}

\begin{proof}
  
\end{proof}

\sk

\begin{problem}[7.4]
  An $R$-module $M$ is said to satisfy the descending chain condition (DCC)
  on submodules if any strictly decreasing chain of submodules of $M$ of
  finite length.
  \begin{enumerate}
  \item[(a)] Show that if $M$ satisfies the DCC, then any nonempty set
    of submodules of $M$ contains a minimal element.
  \item[(b)] Show that $\ell(M)<\infty$ if and only if $M$ satisfies $M$
    satisfies both the ACC and DCC.
  \end{enumerate}
\end{problem}

\begin{proof}
  
\end{proof}

\sk

\begin{problem}[7.5]
  Let
  \[
    R=
    \{
    \left[
      \begin{array}{cc}
        a&b\\
        0&c\\
      \end{array}
    \right]
    :
    a,b\in\bb{R}; c\in\bb{Q}
    \}
  \]
  $R$ is a ring under matrix addition and multiplication. Show that $R$
  satisfies the ACC and DCC on left ideals but neither chain condition
  is valid for right ideals. Thus $R$ is of finite length as a left
  $R$-module, but $\ell(R)=\infty$ as a right $R$-module.
\end{problem}

\begin{proof}
  
\end{proof}

\sk

\begin{problem}[7.11]
  Let $F$ be a field, let $V$ be a finite-dimensional vector space over $F$,
  and let $T\in \text{End}_F(V)$. We shall say that $T$ is semisimple if the
  $F[X]$-module $V_T$ is semisimple. If $A\in M_n(F)$, we shall say that $A$
  is semisimple if the linear transformation $T_A: F^n\rightarrow F^n$
  (multiplication by $A$) is semisimple. Let $\bb{F}_2$ be the field with
  $2$ elements and let $F=\bb{F}_2(Y)$ be the rational function field in
  the indeterminate $Y$, and let $K=F[X]/\langle X^2+Y\rangle$. Since
  $X^2+Y\in F[X]$ is irreducible, $K$ is a field containing
  $F$ as a subfield. Now let
  \[
    A=C(X^2+Y)=
    \left[
      \begin{array}{cc}
        0&Y\\
        1&0\\
      \end{array}
    \right]
    \in M_2(F)
  \]
  Show that $A$ is semisimple when considered in $M_2(F)$ but $A$ is not
  semisimple when considered in $M_2(K)$. Thus, semisimplicity of a matrix
  is not necessarily preserved when one passes to a larger field. However,
  prove that if $L$ is a subfield of the complex numbers $\bb{C}$, then
  $A\in M_n(L)$ is semisimple if and only if it is also semisimple as a complex
  matrix.
\end{problem}

\begin{proof}
  
\end{proof}

\sk

\begin{problem}[7.17]
  \begin{enumerate}
  \item[(a)] Prove that if $R$ is a semisimple ring and $I$ is an ideal,
    then $R/I$ is semisimple.
  \item[(b)] Show (by example) that a subring of a semisimple ring need
    not be semisimple.
  \end{enumerate}
\end{problem}

\begin{proof}
  
\end{proof}

%%%%%%%%%%%%%%%%%%%%%%%%%%%%%%%%%%%%%%%%%%%%%%%%%%%%%%%%%%%%%%%%%%%%%%%%%%%%% 
\end{document}
