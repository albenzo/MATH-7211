\documentclass[10pt]{article}
\usepackage[utf8]{inputenc}
\usepackage{amscd}
\usepackage{amsmath}
\usepackage{amssymb}
\usepackage{amsthm}
\usepackage{listings}
\usepackage{enumerate}

\textwidth=15cm \textheight=22cm \topmargin=0.5cm \oddsidemargin=0.5cm \evensidemargin=0.5cm

\newcommand{\sk}{\vskip 10mm}
\newcommand{\bb}[1]{\mathbb{#1}}
\newcommand{\ra}{\rightarrow}

\theoremstyle{plain}
\newtheorem{problem}{Problem}
\newtheorem{lemma}{Lemma}[problem]

\theoremstyle{remark}
\newtheorem{tpart}{}[problem]
\newtheorem*{ppart}{}

\begin{document}

\begin{problem}
  Show that $x^3 + 3x+1$ is irreducible over $\mathbb{Q}$ and let
  $\theta \in \mathbb{C}$ be a root. Compute $(1 + \theta)(1+\theta+\theta^2)$
  and $\frac{1 + \theta}{1 + \theta + \theta^2}$ in $\mathbb{Q}(\theta)$.
\end{problem}

By the rational roots theorem if rational roots for $x^3+3x+1$ exist
then they must be of the form $\pm 1$. However neither of those are
roots. Thus $x^3+3x+1$ is irreducible over $\bb{Q}$.

Let $\theta\in\bb{C}$ be a root of $x^3+3x+1$.
Then for the expression $(1+\theta)(1+\theta+\theta^2)$ we have:
\begin{align*}
  (1+\theta)(1+\theta+\theta^2) &= 1+2\theta+2\theta^2+\theta^3\\
                &= 2\theta^2-\theta +1+3\theta+\theta^3\\
                &= 2\theta^2-\theta\\
\end{align*}

For the next expression, $\frac{1+\theta}{1+\theta+\theta^2}$, the multiplicative inverse
of the bottom $1+\theta+\theta^2$ is $\frac{3}{7}\theta^2-\frac{2}{7}\theta+\frac{8}{7}$. This
can be found be multiplying $1+\theta+\theta^2$ by $(c+b\theta+c\theta^2)$ and extracting a system
of linear equations. Then we have:
\begin{align*}
  \frac{1+\theta}{1+\theta+\theta^2} &= (1+\theta)\left(\frac{3}{7}\theta^2-\frac{2}{7}\theta+\frac{8}{7}\right)\\
                     &= \frac{3}{7} \theta^{3} + \frac{1}{7} \theta^{2} + \frac{6}{7} \theta + \frac{8}{7}\\
                     &= \frac{1}{7} \theta^{2} - \frac{3}{7} \theta + \frac{5}{7}
\end{align*}

\sk

\begin{problem}
  Let $w = e^{\pi i /6}$ so that $w^{12} = 1$, but $w^k \neq 1$
  for $1 \leq k < 12$. Find the minimal polynomial $m_{w,\mathbb{Q}}(x)$
  and compute $[\mathbb{Q}[w]:\mathbb{Q}]$.
\end{problem}

Begin with the polynomial $x^{12}-1$ which we now that $\omega$ is a root of. This
factors as
\[ x^{12}-1=(x^6-1)(x^6+1) \]
Since $\omega$ of $x^6+1$ and not the other we continue with it. This factors as
\[ x^6+1=(x^2+1)(x^4-x^2+1)\]
As before $x^4-x^2+1$ has $\omega$ as a root and the other does not.
Now we will show that $x^4-x^2+1$ is irreducible.

By the rational root theorem the only possible rational roots are $\pm 1$.
However neither of these are roots. The only way $x^4-x^2+1$ would not
be irreducible is if it were the product of quadratics. Now consider
$x^4-x+1$ as a polynomial with integer coefficients and suppose that

\[x^4-x^2+1 = (ax^2+bx+c)(dx^2+ex+f)=adx^4+(bd+ae)x^3+(af+be+cd)x^2+(bf+ce)x+fc\]

where $a,b,c,d,e\in\bb{Z}$. This gives us the following system of equations:
\begin{align*}
  ad &=1\\
  bd+ae &= 0\\
  af+dc+be &= -1\\
  bf+ce &=0 \\
  fc &= 1\\
\end{align*}

Consider these equations as polynomials in $\bb{C}[a,b,c,d,e,f]$. Then
consider the ideal
\[ \langle ad-1,bd+ae,af+dc+be+1,bf+ce,fc-1\rangle \]
The Gr\"obner basis for this ideal is $\langle 1\rangle$. Since we are in an algebraically
closed field there are no solutions to a set of polynomial equations when
the ideal is the whole ring. Thus $x^4-x^2+1$ is not the product of quadratics
and as such $x^4-x^2+1$ is irreducible.

Therefore $x^4-x^2+1$ is the minimal polynomial $m_{\omega,\bb{Q}}(x)$ and as such
$[\bb{Q}[\omega]:\bb{Q}]=4$.

\begin{problem}
  Compute the minimal polynomial $m_{\alpha,F}(x)$ where
  $\alpha = \sqrt{2} + \sqrt{5}$ and $F$ is each of the following fields:\\
  (a) $\mathbb{Q}$, \hspace{1cm} (b)
  $\mathbb{Q}[\sqrt{5}]$, \hspace{1cm} (c) $\mathbb{Q}[\sqrt{10}]$,
  \hspace{1cm} (d) $\mathbb{Q}[\sqrt{15}]$.
\end{problem}

First note that $\bb{Q}[\alpha]\cong\bb{Q}[\sqrt{2},\sqrt{5}]$ as the latter has a
basis of $(1,\sqrt{2},\sqrt{5},\sqrt{10})$ and we can construct this basis
in $\bb{Q}[\alpha]$. To see this we can construct $\bb{Q}[\sqrt{5}]$ by
\[ \sqrt{2}=\frac{(\alpha^2-7)\alpha-2\alpha}{8}\]
Then we can readily construct the rest.

\begin{enumerate}
\item[(a)] The minimal polynomial is $x^4-14x^2+9$. The fact that this is irreducible
  follows from the tower theorem as $[\bb{Q}[\sqrt{5}]:\bb{Q}]=2$ and
  $[\bb{Q}[\sqrt{5}][\sqrt{2}]:\bb{Q}[\sqrt{5}]]=2$ giving us that
  $[\bb{Q}[\sqrt{5}][\sqrt{2}]:\bb{Q}]=4$. Since we have found a fourth
  degree polynomial that has $\alpha$ as a root we are done.
\item[(b)] The minimal polynomial is $x^2-\sqrt{5}x+2$. As above the irreducibility
  follows from the degree of the extension being two and us finding a polynomial
  with $\alpha$ as a root.
\item[(c)] The minimal polynomial is $x^2-(7+\sqrt{10})$. To show that
  it is irreducible note that $\bb{Q}[\sqrt{10}][\alpha]=\bb{Q}[\sqrt{2},\sqrt{5}]$ because
  as above we have
  \[ \sqrt{2} = \frac{\sqrt{10}\alpha-2\alpha}{8} \]
  By the tower theorem the degree $[\bb{Q}[\sqrt{10}]:\bb{Q}]=2$ as it cannot 1 nor can it
  be 4 as that would imply that $[\bb{Q}[\sqrt{10}][\alpha]:\bb{Q}[\sqrt{10}]]=1$. However since
  $\sqrt{2}\notin\bb{Q}[\sqrt{10}]$ this cannot occur. Thus the degree must be two and since
  we have found a polynomial that has degree 2 with $\alpha$ as a root it must be our minimal polynomial.
\item[(d)] The minimal polynomial will be $x^4-14x^2+9$ as well. The
  fact that it is irreducible will follow from it being irreducible over
  $\bb{Q}$ if $\sqrt{15}$ is not in the span of $\{1,\alpha\}$.

  To show this suppose that 
  \[ \sqrt{15} = a +b\sqrt{2}+c\sqrt{5}+d\sqrt{10} \]
  If we square both sides we get
  \[15 = a^2 + 2\sqrt{2} a b + 2\sqrt{5} a c + 2\sqrt{10} a d
    + 2 b^2 + 2\sqrt{10} b c + 4\sqrt{5} b d + 5 c^2 + 10\sqrt{2} c d + 10 d^2 \]

  This gives us the system of equations
  \begin{align*}
    a^2+2b^2+5c^2+10d^2-15 &= 0\\
    2ab+10cd &= 0\\
    2ac+4bd &= 0\\
    2ad+2bc &= 0\\
  \end{align*}

  This system of equations has no solution over $\bb{Q}$. Thus the degree of
  the extension $\bb{Q}[\sqrt{15}]$ by $\alpha$ is the same as $[\bb{Q}[\alpha]:\bb{Q}]=4$.
  Since we have a monic polynomial with $\alpha$ as the root that is of the proper degree it
  must be the minimal polynomial.
\end{enumerate}

\sk

\begin{problem}
  Compute the minimal polynomial $m_{\alpha, \mathbb{Q}}(x)$
  where $\alpha = \sqrt{2} + \sqrt[3]{5}$.
\end{problem}

Consider the polynomial $f(x)=x^{6} - 6 x^{4} - 10 x^{3} + 12 x^{2} - 60 x + 17$. Then $f(\alpha)=0$.
Now we wish to show that $f(x)$ is irreducible. By the rational roots theorem
if any rational roots exist then they will be of the form $\pm 17$ neither of which are
roots. Therefore if $f(x)$ is reducible it will either be the product of two cubics or
the product of a quartic and a quadratic.

Suppose that $f(x)$ was the product of two cubics. Then we would have
\[ x^{6} - 6 x^{4} - 10 x^{3} + 12 x^{2} - 60 x + 17 = (x^3+ax^2+bx+c)(x^3+dx^2+ex+f) \]
If we multiply out the latter terms we can extract the system of equations
\begin{align*}
  a+d &=0\\
  ad+b+e +6 &= 0\\
  ad+be+c+f+10&= 0\\
  af+eb+cd-12&= 0\\
  bf+ec+60&= 0\\
  cf - 17 &= 0\\
\end{align*}

If we consider the ideal
\[\langle a+d,,ad+b+e,ad+be+c+f+10,af+eb+cd-12,bf+ec+60,cf-17\rangle\subset \bb{C}[a,b,c,d,e,f]\]
the Gr\"obner basis of this ideal is $\langle 1\rangle$ which implies that there are no solutions
to the equation. Thus $f(x)$ cannot be the product of two cubics.

Similarly suppose that $f(x)$ was the product of a quartic and a quadratic.
Then
\[ x^{6} - 6 x^{4} - 10 x^{3} + 12 x^{2} - 60 x + 17 = (x^4+ax^3+bx^2+cx+d)(x^2+ex+f)\]
  \[=x^{6} + \left(a + e\right) x^{5} + \left(a e + b + f\right) x^{4} + \left(b e + a f + c\right) x^{3} + \left(c e + b f + d\right) x^{2} + \left(d e + c f\right) x + d f \]

This gives us the system of equations
\begin{align*}
  a+e&= 0\\
  ae+b+f+6&= 0\\
  be+af+c+10&= 0\\
  ce+bf+d-12&= 0\\
  de+cf+60&= 0\\
  df -17&= 0\\
\end{align*}

As before consider the ideal
\[ \langle d f - 17, d e + c f + 60, c e + b f + d - 12, b e + a f + c + 10, a e + b + f + 6\rangle
  \subset \bb{C}[a,b,c,d,e,f] \]

The Gr\"obner basis for this ideal is $\langle 1\rangle$ which implies that there are no solutions.
Therefore $f(x)$ cannot be expressed as the product of a quartic and a quadratic.

Therefore $f(x)$ is irreducible and as such is in fact the minimal polynomial for $\alpha$.

\sk

\begin{problem}
  If $K$ is a field extension of the field of $F$ and
  $\alpha \in K$ has a minimal polynomial $f(x) \in F[x]$ of odd degree,
  prove that $F(\alpha) = F(\alpha^2)$. Determine whether the condition
  on $f(x)$ is necessary for $F(\alpha) = F(\alpha^2)$.
\end{problem}

\begin{proof}
  First note that $F\subset F(\alpha^2)\subset F(\alpha)$ as $\alpha^2\in F(\alpha)$. Then using the tower theorem
  we know that
  \[ [F(\alpha):F] = [F(\alpha):F(\alpha^2)][F(\alpha^2):F(\alpha)] \]
  By assumption we know that $[F(\alpha):F]$ is odd. Moreover we have that
  $[F(\alpha):F(\alpha^2)]=[F(\alpha^2)(\alpha):F(\alpha^2)]$. The degree $[F(\alpha^2)(\alpha):F(\alpha^2)]$ will be
  less than or equal to 2 since $x^2-\alpha^2$ has $\alpha$ as root. However it cannot
  be 2 since this would contradict $[F(\alpha):F]$ being odd. As such the
  minimal polynomial for $F(\alpha)$ over $F(\alpha^2)$ must be linear, which implies
  that $\alpha\in F(\alpha^2)$ and therefore $F(\alpha)=F(\alpha^2)$.
\end{proof}

The condition is not necessary. Consider the polynomial
\[(x-e^{2\pi i/3})(x-e^{4\pi i/3})=x^2+x+1\]
Note that both roots are squares of each other making their extensions equal.
However $x^2+x+1$ is irreducible making the degree even.

\sk

\begin{problem}{6}
  Let $K$ be an extension field of $F$ that is algebraic over $F$.
  Show that any subring $R$ of $K$ which contains $F$, i.e.,
  $F \subseteq R \subseteq K$, is a field. Hence, prove that any subring of
  a finite dimensional extension field $K/F$ containing $F$ is a subfield.
\end{problem}

\begin{proof}
  As $R$ is a subring of the field $K$ we know that it fulfills all the
  properties of a field except possibly multiplicative inverses. Let
  $r\in R$. Since $K$ is algebraic over $F$ there is an irreducible
  polynomial $f(x)=\sum_0^na_ix^i$ where $r$ is a root of $f(x)$. Consider
  \[f(r)=\sum_0^na_ir^i=0 \]
  The constant term will be nonzero as $f$ is irreducible. As such
  move $a_0$ to the right and factor to get
  \[r\sum_1^na_ir^{i-1} = -a_0\]
  Since $-a_0\in F$ it has an inverse. Thus
  \[r\cdot\frac{1}{-a_0}\sum_1^na_ir^{i-1}=1 \]
  and $\frac{1}{-a_0}\sum_1^na_ir^{i-1}$ is the multiplicative inverse to $r$.
  This implies that $R$ must be a field.

  Since finite dimensional field extension are algebraic it follows that any
  subring of a finite dimensional field extension $K/f$ containing $F$ is a
  subfield.
\end{proof}

\sk

\begin{problem}
  Suppose that $K = F(\alpha)$ is a finite simple extension of the field $F$.
  Define an $F$-linear transformation
  $T_\alpha: K \to K$ by $T_\alpha (\beta) = \alpha\beta$ for all $\beta \in K$.
  Show that the minimal polynomial of $\alpha$ over $F$ is the characteristic
  polynomial of $T_\alpha$, that is
  \begin{equation*}
    m_{\alpha, F}(x) = det(xI - T_\alpha).
  \end{equation*}
\end{problem}

\begin{proof}
  First let $n:=[F(\alpha):F]$ and let $\sum_0^nr_ix_i$ be the minimal polynomial for
  $F(\alpha)$. Let $\beta=\sum_0^{n-1}c_i\alpha^i\in F(\alpha)$ where $c_i\in F$. Then
  \[T_\alpha(\beta)=\sum_0^{n-1}c_i\alpha^{i+1} \]
  Using minimal polynomial to remove the $\alpha^n$ we can simplify the expression to
  \[ T_\alpha(\beta)=\sum_0^{n-2}(c_i-c_{n-1}r_{i+1})\alpha^{i+1}-c_{n-1}r_0 \]
  which gives us that the matrix for $T_\alpha$ is
  \[
    [T_\alpha]=
    \left(
      \begin{array}{ccccc}
        0&0&\cdots&0&-r_0\\
        1&0&\cdots&0&-r_1\\
        0&1&\cdots&0&-r_2\\
        \vdots&\vdots&\ddots&\vdots&\vdots\\
        0&0&0&1&-r_{n-1}\\
      \end{array}
    \right)
  \]
  Note that this matrix is a companion matrix for the
  rational canonical form which implies that the determinant of
  \[
    xI-[T_\alpha] =
    \left(
      \begin{array}{ccccc}
        x&0&\cdots&0&-r_0\\
        1&x&\cdots&0&-r_1\\
        0&1&\cdots&0&-r_2\\
        \vdots&\vdots&\ddots&\vdots&\vdots\\
        0&0&0&1&x-r_{n-1}\\
      \end{array}
    \right)
  \]
  is precisely $\sum_0^{n}r_ix^i$ completing the proof.
  
\end{proof}

%%%%%%%%%%%%%%%%%%%%%%%%%%%%%%%%%%%%%%%%%%%%%%%%%%%%%%%%%%%%%%%%%%%%%%%%%%%%%
\end{document}
