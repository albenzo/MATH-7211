\documentclass[10pt]{article}
\usepackage[utf8]{inputenc}
\usepackage{amscd}
\usepackage{amsmath}
\usepackage{amssymb}
\usepackage{amsthm}
\usepackage{listings}
\usepackage{enumerate}

\textwidth=15cm \textheight=22cm \topmargin=0.5cm \oddsidemargin=0.5cm \evensidemargin=0.5cm

\newcommand{\sk}{\vskip 10mm}
\newcommand{\bb}[1]{\mathbb{#1}}
\newcommand{\ra}{\rightarrow}

\theoremstyle{plain}
\newtheorem{problem}{Problem}
\newtheorem{lemma}{Lemma}[problem]

\theoremstyle{remark}
\newtheorem{tpart}{}[problem]
\newtheorem*{ppart}{}

\begin{document}

\begin{problem}
  Show that $x^3 + 3x+1$ is irreducible over $\mathbb{Q}$ and let
  $\theta \in \mathbb{C}$ be a root. Compute $(1 + \theta)(1+\theta+\theta^2)$
  and $\frac{1 + \theta}{1 + \theta + \theta^2}$ in $\mathbb{Q}(\theta)$.
\end{problem}

By the rational roots theorem if rational roots for $x^3+3x+1$ exist
then they must be of the form $\pm 1$. However neither of those are
roots. Thus $x^3+3x+1$ is irreducible over $\bb{Q}$.

Let $\theta\in\bb{C}$ be a root of $x^3+3x+1$.
Then for the expression $(1+\theta)(1+\theta+\theta^2)$ we have:
\begin{align*}
  (1+\theta)(1+\theta+\theta^2) &= 1+2\theta+2\theta^2+\theta^3\\
                &= 2\theta^2-\theta +1+3\theta+\theta^3\\
                &= 2\theta^2-\theta\\
\end{align*}

For the next expression, $\frac{1+\theta}{1+\theta+\theta^2}$, the multiplicative inverse
of the bottom $1+\theta+\theta^2$ is $\frac{3}{7}\theta^2-\frac{2}{7}\theta+\frac{8}{7}$. This
can be found be multiplying $1+\theta+\theta^2$ by $(c+b\theta+c\theta^2)$ and extracting a system
of linear equations. Then we have:
\begin{align*}
  \frac{1+\theta}{1+\theta+\theta^2} &= (1+\theta)\left(\frac{3}{7}\theta^2-\frac{2}{7}\theta+\frac{8}{7}\right)\\
                     &= \frac{3}{7} \theta^{3} + \frac{1}{7} \theta^{2} + \frac{6}{7} \theta + \frac{8}{7}\\
                     &= \frac{1}{7} \theta^{2} - \frac{3}{7} \theta + \frac{5}{7}
\end{align*}

\sk

\begin{problem}
  Let $w = e^{\pi i /6}$ so that $w^{12} = 1$, but $w^k \neq 1$
  for $1 \leq k < 12$. Find the minimal polynomial $m_{w,\mathbb{Q}}(x)$
  and compute $[\mathbb{Q}[w]:\mathbb{Q}]$.
\end{problem}

Begin with the polynomial $x^{12}-1$ which we now that $\omega$ is a root of. This
factors as
\[ x^{12}-1=(x^6-1)(x^6+1) \]
Since $\omega$ of $x^6+1$ and not the other we continue with it. This factors as
\[ x^6+1=(x^2+1)(x^4-x^2+1)\]
As before $x^4-x^2+1$ has $\omega$ as a root and the other does not.
Now we will show that $x^4-x^2+1$ is irreducible.

By the rational root theorem the only possible rational roots are $\pm 1$.
However neither of these are roots. The only way $x^4-x^2+1$ would not
be irreducible is if it were the product of quadratics. Now consider
$x^4-x+1$ as a polynomial with integer coefficients and suppose that

\[x^4-x^2+1 = (ax^2+bx+c)(dx^2+ex+f)=adx^4+(bd+ae)x^3+(af+be+cd)x^2+(bf+ce)x+fc\]

where $a,b,c,d,e\in\bb{Z}$. This gives us the following system of equations:
\begin{align*}
  ad &=1\\
  bd+ae &= 0\\
  af+dc+be &= -1\\
  bf+ce &=0 \\
  fc &= 1\\
\end{align*}

Consider these equations as polynomials in $\bb{C}[a,b,c,d,e,f]$. Then
consider the ideal
\[ \langle ad-1,bd+ae,af+dc+be+1,bf+ce,fc-1\rangle \]
The Gr\"obner basis for this ideal is $\langle 1\rangle$. Since we are in an algebraically
closed field there are no solutions to a set of polynomial equations when
the ideal is the whole ring. Thus $x^4-x^2+1$ is not the product of quadratics
and as such $x^4-x^2+1$ is irreducible.

Therefore $x^4-x^2+1$ is the minimal polynomial $m_{\omega,\bb{Q}}(x)$ and as such
$[\bb{Q}[\omega]:\bb{Q}]=4$.

\begin{problem}
  Compute the minimal polynomial $m_{\alpha,F}(x)$ where
  $\alpha = \sqrt{2} + \sqrt{5}$ and $F$ is each of the following fields:\\
  (a) $\mathbb{Q}$, \hspace{1cm} (b)
  $\mathbb{Q}[\sqrt{5}]$, \hspace{1cm} (c) $\mathbb{Q}[\sqrt{10}]$,
  \hspace{1cm} (d) $\mathbb{Q}[\sqrt{15}]$.
\end{problem}

\begin{enumerate}
\item[(a)]
\item[(b)]
\item[(c)]
\item[(d)]
\end{enumerate}

\sk

\begin{problem}
  Compute the minimal polynomial $m_{\alpha, \mathbb{Q}}(x)$
  where $\alpha = \sqrt{2} + \sqrt[3]{5}$.
\end{problem}

Consider the polynomial $f(x)=x^{6} - 6 x^{4} - 10 x^{3} + 12 x^{2} - 60 x + 17$. Then $f(\alpha)=0$.
Now we wish to show that $f(x)$ is irreducible. By the rational roots theorem
if any rational roots exist then they will be of the form $\pm 17$ neither of which are
roots. Therefore if $f(x)$ is reducible it will either be the product of two cubics or
the product of a quartic and a quadratic.

Suppose that $f(x)$ was the product of two cubics. Then we would have
\[ x^{6} - 6 x^{4} - 10 x^{3} + 12 x^{2} - 60 x + 17 = (x^3+ax^2+bx+c)(x^3+dx^2+ex+f) \]
If we multiply out the latter terms we can extract the system of equations
\begin{align*}
  a+d &=0\\
  ad+b+e +6 &= 0\\
  ad+be+c+f+10&= 0\\
  af+eb+cd-12&= 0\\
  bf+ec+60&= 0\\
  cf - 17 &= 0\\
\end{align*}

If we consider the ideal
\[\langle a+d,,ad+b+e,ad+be+c+f+10,af+eb+cd-12,bf+ec+60,cf-17\rangle\subset \bb{C}[a,b,c,d,e,f]\]
the Gr\"obner basis of this ideal is $\langle 1\rangle$ which implies that there are no solutions
to the equation. Thus $f(x)$ cannot be the product of two cubics.

Similarly suppose that $f(x)$ was the product of a quartic and a quadratic.
Then
\[ x^{6} - 6 x^{4} - 10 x^{3} + 12 x^{2} - 60 x + 17 = (x^4+ax^3+bx^2+cx+d)(x^2+ex+f)\]
  \[=x^{6} + \left(a + e\right) x^{5} + \left(a e + b + f\right) x^{4} + \left(b e + a f + c\right) x^{3} + \left(c e + b f + d\right) x^{2} + \left(d e + c f\right) x + d f \]

This gives us the system of equations
\begin{align*}
  a+e&= 0\\
  ae+b+f+6&= 0\\
  be+af+c+10&= 0\\
  ce+bf+d-12&= 0\\
  de+cf+60&= 0\\
  df -17&= 0\\
\end{align*}

As before consider the ideal
\[ \langle d f - 17, d e + c f + 60, c e + b f + d - 12, b e + a f + c + 10, a e + b + f + 6\rangle
  \subset \bb{C}[a,b,c,d,e,f] \]

The Gr\"obner basis for this ideal is $\langle 1\rangle$ which implies that there are no solutions.
Therefore $f(x)$ cannot be expressed as the product of a quartic and a quadratic.

Therefore $f(x)$ is irreducible and as such is in fact the minimal polynomial for $\alpha$.

\sk

\begin{problem}
  If $K$ is a field extension of the field of $F$ and
  $\alpha \in K$ has a minimal polynomial $f(x) \in F[x]$ of odd degree,
  prove that $F(\alpha) = F(\alpha^2)$. Determine whether the condition
  on $f(x)$ is necessary for $F(\alpha) = F(\alpha^2)$.
\end{problem}

\begin{proof}
  
\end{proof}

\sk

\begin{problem}{6}
  Let $K$ be an extension field of $F$ that is algebraic over $F$.
  Show that any subring $R$ of $K$ which contains $F$, i.e.,
  $F \subseteq R \subseteq K$, is a field. Hence, prove that any subring of
  a finite dimensional extension field $K/F$ containing $F$ is a subfield.
\end{problem}

\begin{proof}
  
\end{proof}

\sk

\begin{problem}{7}
  Suppose that $K = F(\alpha)$ is a finite simple extension of the field $F$.
  Define an $F$-linear transformation
  $T_\alpha: K \to K$ by $T_\alpha (\beta) = \alpha\beta$ for all $\beta \in K$.
  Show that the minimal polynomial of $\alpha$ over $F$ is the characteristic
  polynomial of $T_\alpha$, that is
  \begin{equation*}
    m_{\alpha, F}(x) = det(xI - T_\alpha).
  \end{equation*}
\end{problem}

\begin{proof}
  
\end{proof}

%%%%%%%%%%%%%%%%%%%%%%%%%%%%%%%%%%%%%%%%%%%%%%%%%%%%%%%%%%%%%%%%%%%%%%%%%%%%%
\end{document}
