\documentclass[10pt]{article}
\usepackage[utf8]{inputenc}
\usepackage{amscd}
\usepackage{amsmath}
\usepackage{amssymb}
\usepackage{amsthm}
\usepackage{listings}
\usepackage{enumerate}
\usepackage[all,cmtip]{xy}

\textwidth=15cm \textheight=22cm \topmargin=0.5cm \oddsidemargin=0.5cm \evensidemargin=0.5cm

\newcommand{\sk}{\vskip 10mm}
\newcommand{\bb}[1]{\mathbb{#1}}
\newcommand{\ra}{\rightarrow}

\theoremstyle{plain}
\newtheorem{problem}{Problem}
\newtheorem{lemma}{Lemma}[problem]

\theoremstyle{remark}
\newtheorem{tpart}{}[problem]
\newtheorem*{ppart}{}

\begin{document}

\begin{problem}
  Let $G=C_2=\langle a:a^2=1\rangle$, and let $V=F^2$ (where $F$ is a field).
  For $(\alpha,\beta)\in V$, define the action of $G$ on $V$ by
  $1(\alpha,\beta)=(\alpha,\beta)$ and $a(\alpha,\beta)=(\beta,\alpha)$, and
  extend by linearity to make $V$ into an $FG$-module. Find all $FG$-submodules
  of $V$.
\end{problem}

First note that the submodules correspond to subspaces that are invariant
under the action of $FG$. In this case, due to the transposition, the only
such subspaces are $\{0\},V$, $\{(\alpha,\alpha)\in V\}$, and $\{(\alpha,\beta)\in V|\alpha+\beta=0\}$.

\sk

\begin{problem}
  If $G=C_2\times C_2=\langle a,b:a^2=b^2=1,ab=ba\rangle$, write the real group
  ring $\bb{R}G$ as a direct sum of $\bb{R}G$-submodules, each of which
  is $1$-dimensional over $\bb{R}$.
\end{problem}

Treat this as a vector space over $\bb{R}$ with basis $(e_1,e_2,e_3,e_4)$ where
the corresponding representation acts as
\[
  \varphi(a)=(e_2,e_1,e_3,e_r),\varphi(b)=(e_1,e_2,e_4,e_3)
\]

Then we can express this as the direct sum of the four following 1-dimensional
subspaces that are invariant under the action:
\begin{itemize}
\item $\{(xe_1+xe_2)|x\in\bb{R}\}$
\item $\{(xe_3+xe_4)|x\in\bb{R}\}$
\item $\{(xe_1+ye_2)|x+y=0\}$
\item $\{(xe_3+ye_4)|x+y=0\}$
\end{itemize}

\sk

\begin{problem}
  Let $G=D_{12}=\langle a,b:a^6=b^2=1,b^{-1}ab=a^{-1}\rangle$.
  Define matrices $A,B,C,D$ over $\bb{C}$ by
  \[
    A
    =
    \left[
      \begin{array}{cc}
        e^{i\pi/3} & 0\\
        0 & e^{-i\pi/3}\\
      \end{array}
    \right],
    B
    =
    \left[
      \begin{array}{cc}
        0 & 1\\
        1 & 0\\
      \end{array}
    \right],
    C
    =
    \left[
      \begin{array}{cc}
        1/2 & \sqrt{3}/2\\
        -\sqrt{3}/2 & 1/2\\
      \end{array}
    \right],
    D
    =
    \left[
      \begin{array}{cc}
        1 & 0\\
        0 & -1\\
      \end{array}
    \right]
  \]
  \begin{enumerate}
  \item[(a)] Verify that each of the functions
    $\rho_k:G\rightarrow\text{GL}(2,\bb{C})\ (k=1,2,3,4)$, given by
    $(i)\ \rho_1(a^r b^s)=A^r B^s,\quad
    (ii)\ \rho_2(a^r b^s)= A^{3r}(-B)^s,\quad
    (iii)\ \rho_3(a^r b^s)=(-A)^r B^s,\quad
    (iv)\ \rho_4(a^r b^s)=C^r D^s$
    for $0\leq r\leq 5, 0\leq s\leq 1$, is a representation of $G$.
  \item[(b)] Which of the representations $\rho_k$ are faithful?
  \item[(c)] Which of the representations are equivalent?
  \item[(d)] Which are irreducible?
  \end{enumerate}
\end{problem}

\begin{enumerate}
\item[(a)] To show that the $\rho_k$ are representations we will verify that
  the relations of $D_{12}$ are fulfilled after applying $\rho_k$.

  For $\rho_1$:
  \[
    \rho_1(a)^6=
    \left(
      \begin{array}{rr}
        {\left(\frac{1}{2} i \, \sqrt{3} + \frac{1}{2}\right)}^{6} & 0 \\
        0 & {\left(-\frac{1}{2} i \, \sqrt{3} + \frac{1}{2}\right)}^{6}
      \end{array}
    \right)
    =
    I_2
    =
    \left(
      \begin{array}{rr}
        0 & 1 \\
        1 & 0
      \end{array}
    \right)^2
    =
    \rho_1(b)^2
  \]

  \[
    \rho_1(b)^{-1}\rho_1(a)\rho_1(b)
    =
    \left(
      \begin{array}{rr}
        0 & 1 \\
        1 & 0
      \end{array}
    \right)
    \left(
      \begin{array}{rr}
        {\left(\frac{1}{2} i \, \sqrt{3} + \frac{1}{2}\right)}^{6} & 0 \\
        0 & {\left(-\frac{1}{2} i \, \sqrt{3} + \frac{1}{2}\right)}^{6}
      \end{array}
    \right)
    \left(
      \begin{array}{rr}
        0 & 1 \\
        1 & 0
      \end{array}
    \right)
    =
    \left(
      \begin{array}{rr}
        -\frac{1}{2} i \, \sqrt{3} + \frac{1}{2} & 0 \\
        0 & \frac{1}{2} i \, \sqrt{3} + \frac{1}{2}
      \end{array}
    \right)
  \]

  For $\rho_2$:
    \[
    \rho_2(a)^6=
    \left(\begin{array}{rr}
{\left(\frac{1}{2} i \, \sqrt{3} + \frac{1}{2}\right)}^{18} & 0 \\
0 & {\left(-\frac{1}{2} i \, \sqrt{3} + \frac{1}{2}\right)}^{18}
\end{array}\right)
    =
    I_2
    =
    \left(
      \begin{array}{rr}
        0 & -1 \\
        -1 & 0
      \end{array}
    \right)^2
    =
    \rho_2(b)^2
  \]

  \[\hspace*{-2.4cm}
    \rho_2(b)^{-1}\rho_2(a)\rho_2(b)
    =
    \left(
      \begin{array}{rr}
        0 & -1 \\
        -1 & 0
      \end{array}
    \right)
\left(\begin{array}{rr}
{\left(\frac{1}{2} i \, \sqrt{3} + \frac{1}{2}\right)}^{3} & 0 \\
0 & {\left(-\frac{1}{2} i \, \sqrt{3} + \frac{1}{2}\right)}^{3}
\end{array}\right)
    \left(
      \begin{array}{rr}
        0 & -1 \\
        -1 & 0
      \end{array}
    \right)
    =
    \left(\begin{array}{rr}
{\left(-\frac{1}{2} i \, \sqrt{3} + \frac{1}{2}\right)}^{3} & 0 \\
0 & {\left(\frac{1}{2} i \, \sqrt{3} + \frac{1}{2}\right)}^{3}
\end{array}\right)
  \]

  
  For $\rho_3$:
    \[
      \rho_3(a)^6=
      \left(\begin{array}{rr}
{\left(-\frac{1}{2} i \, \sqrt{3} - \frac{1}{2}\right)}^{6} & 0 \\
0 & {\left(\frac{1}{2} i \, \sqrt{3} - \frac{1}{2}\right)}^{6}
\end{array}\right)
    =
    I_2
    =
    \left(
      \begin{array}{rr}
        0 & 1 \\
        1 & 0
      \end{array}
    \right)^2
    =
    \rho_3(b)^2
  \]

  \[
    \rho_3(b)^{-1}\rho_3(a)\rho_3(b)
    =
    \left(
      \begin{array}{rr}
        0 & 1 \\
        1 & 0
      \end{array}
    \right)
    \left(\begin{array}{rr}
-\frac{1}{2} i \, \sqrt{3} - \frac{1}{2} & 0 \\
0 & \frac{1}{2} i \, \sqrt{3} - \frac{1}{2}
\end{array}\right)
    \left(
      \begin{array}{rr}
        0 & 1 \\
        1 & 0
      \end{array}
    \right)
    =
    \left(\begin{array}{rr}
\frac{1}{2} i \, \sqrt{3} - \frac{1}{2} & 0 \\
0 & -\frac{1}{2} i \, \sqrt{3} - \frac{1}{2}
\end{array}\right)
  \]

  For $\rho_4$:
    \[
    \rho_4(a)^6=
\left(\begin{array}{rr}
\frac{1}{2} & \frac{1}{2} \, \sqrt{3} \\
-\frac{1}{2} \, \sqrt{3} & \frac{1}{2}
\end{array}\right)^6
    =
    I_2
    =
    \left(
      \begin{array}{rr}
        1 & 0 \\
        0 & -1
      \end{array}
    \right)^2
    =
    \rho_4(b)^2
  \]

  \[
    \rho_4(b)^{-1}\rho_4(a)\rho_4(b)
    =
    \left(
      \begin{array}{rr}
        1 & 0 \\
        0 & -1
      \end{array}
    \right)
    \left(\begin{array}{rr}
\frac{1}{2} & \frac{1}{2} \, \sqrt{3} \\
-\frac{1}{2} \, \sqrt{3} & \frac{1}{2}
\end{array}\right)
    \left(
      \begin{array}{rr}
        1 & 0 \\
        0 & -1
      \end{array}
    \right)
    =
    \left(\begin{array}{rr}
\frac{1}{2} & -\frac{1}{2} \, \sqrt{3} \\
\frac{1}{2} \, \sqrt{3} & \frac{1}{2}
\end{array}\right)
  \]

\item[(b)] We can see that both $\rho_1$ and $\rho_4$ are faithful
  as the order $A$ and $D$ are both 6. However $\rho_2$ and $\rho_3$
  are not faithful as the order of $\rho_2(a)$ is $2$ and the order of
  $\rho_3(a)$ is $3$.
\item[(c)] To start $\rho_2$ and $\rho_3$ are not equivalent to either of the
  other two, or each other for that matter, due to the orders of the elements
  that $a$ is sent to.

  However we can see that $\rho_1$ and $\rho_4$ are equivalent as $A$ and $C$ have the
  same eigenvalues ($e^{i*\pi/3},e^{-i\pi/3}$) and so do matrices $B$ and $D$ ($1,-1$).
\item[(d)] We used the eigenvectors for the matrices involved in $\rho_1$. Moreover
  note that the eigenvalues are distinct and the eigenspaces are distinct. This implies
  that $\rho_1$ is irreducible. We then get that $\rho_4$ is irreducible by equivalence.

  Similarly the eigenvalues for $-A$ are
  $\left[-\frac{1}{2} i \, \sqrt{3} - \frac{1}{2}, \frac{1}{2} i \, \sqrt{3} - \frac{1}{2}\right]$.
  Since the eigenvalues for $-A$ and $B$ are distinct and the eigenspaces are
  distinct we have that $\rho_3$ is irreducible.

  However the eigenvalues for $-B$ are $[-1,1]$ and for $A^3$ they are $[-1,-1]$. Since
  the eigenvalues are not distinct the eigenspaces also will not. This implies that there
  is a non-trivial subrepresentation.
\end{enumerate}

\sk

\begin{problem}
  Find the missing row in the following character table:
  \[
    \begin{array}{c|ccccc}
      \text{Order of the conjugacy class} & (1) & (3) & (6) & (6) & (8)\\
      \text{Conjugacy class} & Cl(1) & Cl(a) & Cl(b) & Cl(c) & Cl(d)\\
      \hline
      \chi_1 & 1 & 1 & 1 & 1 & 1\\
      \chi_2 & 1 & 1 & -1 & -1 & 1\\
      \chi_3 & 3 & -1 & 1 & -1 & 0\\
      \chi_4 & 3 & -1 & -1 & 1 & 0\\
      \chi_5 &&&&&\\
    \end{array}
  \]
\end{problem}

First note that the order of the group is 24. Then using the second
orthogonality relation from Dummit ($\sum_{i=1}^r\chi_i(x)\overline{\chi_i(y)}$) we
know what each column with itself should be the size of the centralizer.
This gives us
\begin{align*}
  24 - (1+1+9+9) &= 4\\
  8 - (1+1+1+1) &= 4\\
  4 - (1+1+1+1) &= 0\\
  4 - (1+1+1+1) &= 0\\
  3 - (1+1) &= 1
\end{align*}

Then we can use the inner product $\langle \chi_1,\chi_5\rangle=0$ to get the signs correct.
When we fill in our table we get

\[
  \begin{array}{c|ccccc}
    \text{Order of the conjugacy class} & (1) & (3) & (6) & (6) & (8)\\
    \text{Conjugacy class} & Cl(1) & Cl(a) & Cl(b) & Cl(c) & Cl(d)\\
    \hline
    \chi_1 & 1 & 1 & 1 & 1 & 1\\
    \chi_2 & 1 & 1 & -1 & -1 & 1\\
    \chi_3 & 3 & -1 & 1 & -1 & 0\\
    \chi_4 & 3 & -1 & -1 & 1 & 0\\
    \chi_5 & 2 & 2 & 0 & 0 & -1\\
  \end{array}
\]

\sk

\begin{problem}
  The character table of $S_3$ is
  \[
    \begin{array}{c|ccc}
      \text{Conjugacy class} & Cl(1) & Cl((1\ 2)) & Cl((1\ 2\ 3))\\
      \hline
      \chi_1 & 1 & 1 & 1\\
      \chi_2 & 1 & -1 & 1\\
      \chi_3 & 2 & 0 & -1\\
    \end{array}
  \]
  Let $\phi$ be a character such that
  $\phi(1)=5,\phi((1\ 2))=1,\phi((1\ 2\ 3))=2$.
  \begin{enumerate}
  \item[(a)] Compute the inner products
    $\langle \phi,\chi_1\rangle,\langle \phi,\chi_2\rangle$,
    and $\langle \phi,\chi_3\rangle$.
  \item[(b)] Write the character $\phi$ as a linear combination of
    $\chi_1,\chi_2,\chi_3$.
  \end{enumerate}
\end{problem}

\begin{enumerate}
\item[(a)] If we calculate the inner products we get
  \[
    \langle \phi,\chi_1\rangle=2, \langle\phi,\chi_2\rangle=1, \langle\phi,\chi_3\rangle=1
  \]
\item[(b)] $\phi=2\chi_1+\chi_2+\chi_3$
\end{enumerate}


Various bits of code that I used to help get a
handle on some of this material.
\begin{verbatim}
A = Matrix(SR,[[e^(i*pi/3),0],[0,e^(-i*pi/3)]])
B = Matrix(SR,[[0,1],[1,0]])
C = Matrix(SR,[[1/2,sqrt(3)/2],[-sqrt(3)/2,1/2]])
D = Matrix(SR,[[1,0],[0,-1]])
G = DihedralGroup(6)
r,s = G.gens()

def p1(g):
    if g in [r^i for i in range(0,6)]:
        return A^([r^i for i in range(0,6)].index(g))
    elif g in [r^i*s for i in range(0,6)]:
        return A^([r^i*s for i in range(0,6)].index(g))*B

def p2(g):
    if g in [r^i for i in range(0,6)]:
        return A^(3*[r^i for i in range(0,6)].index(g))
    elif g in [r^i*s for i in range(0,6)]:
        return A^(3*[r^i*s for i in range(0,6)].index(g))*(-B)

def p3(g):
    if g in [r^i for i in range(0,6)]:
        return (-A)^([r^i for i in range(0,6)].index(g))
    elif g in [r^i*s for i in range(0,6)]:
        return (-A)^([r^i*s for i in range(0,6)].index(g))*B

def p4(g):
    if g in [r^i for i in range(0,6)]:
        return C^([r^i for i in range(0,6)].index(g))
    elif g in [r^i*s for i in range(0,6)]:
        return C^([r^i*s for i in range(0,6)].index(g))*D

def orth1(x1,x2,G):
    return 1/len(G) * sum([x1(g)*conjugate(x2(g)) for g in G])

def orth2(x,y,Cs):
    return sum([c(x)*conjugate(c(y)) for c in Cs])

S3 = SymmetricGroup(3)
cl1 = S3(()).conjugacy_class()
cl12 = S3((1,2)).conjugacy_class()
cl123 = S3((1,2,3)).conjugacy_class()

def x1(g):
    return 1

def x2(g):
    if g in cl1:
        return 1
    if g in cl12:
        return -1
    if g in cl123:
        return 1

def x3(g):
    if g in cl1:
        return 2
    if g in cl12:
        return 0
    if g in cl123:
        return -1

def phi(g):
    if g in cl1:
        return 5
    if g in cl12:
        return 1
    if g in cl123:
        return 2
\end{verbatim}
%%%%%%%%%%%%%%%%%%%%%%%%%%%%%%%%%%%%%%%%%%%%%%%%%%%%%%%%%%%%%%%%%%%%%%%%%%%%%
\end{document}
